\chapter{使用泊松做近似}

\section{非均匀奖券收集(Coupon Collector)问题}

我们\href{https://shuiyuan.sjtu.edu.cn/t/topic/316193}{\underline{以前}}讲过奖券收集问题。

\begin{leftbarquote}
    \quad \quad 考虑玩一个抽卡手游。现在总共有 $n$ 种不同类型的卡,每一抽可以均匀的得到其中一种。现在想问平均要抽多少次,可以集齐一套,即 $n$ 种卡每种至少一张。
\end{leftbarquote}

如果每一次抽卡每一种类型的卡出现的概率都是等概率 $\frac{1}{n}$,那么我们期望需要抽 $n H_n$ 次才能收集到所有类型的卡,其中 $H_n = \sum_{k=1}^n \frac{1}{k}\overset{n\to\infty}{\longrightarrow} n\log n+\gamma n$ 是调和级数。

在实际中,手游公司往往会对每一种卡有稀有度的设定,比如,对于每一抽,第 $i$ 种卡被开出来的概率是 $p_i$ ( $\sum_{i=1}^n p_i=1$ )。那么,在这样一种设定下,集齐一套平均要抽多少次呢?稍微想一下就会明白,因为现在每种卡不再有对称性,我们之前基于期望的线性性的简单技巧不再有效了。

令 $N_i$ 表示第一次获得第 $i$ 种卡片需要抽卡的次数,那么 $N_i$ 服从参数为 $p_i$ 的几何分布。令 $N$ 表示收集到所有 $n$ 种类型卡片所需的抽卡次数,那么 $N = \max_{i \in [n]} N_i$。 

我们问题便是计算 $\E{N}$。然而,由于 $N_i$ 之间不是独立的,直接计算 $\E{N}$ 并不容易。

\subsection{泊松抽卡法}

我们可以脑补如下一个抽卡的方式,和我们要研究的问题是等价的:

\begin{leftbarquote}
    \quad \quad 玩家在一个线下的商店柜台购买卡包进行抽卡。每一分钟过来一位顾客,进行一次抽卡。现在问平均第几位顾客(也就是第几分钟)抽完之后,前面所有的顾客抽的卡放在一起能够集齐所有 $n$ 种。
\end{leftbarquote}



我们现在稍稍修改上面这个场景,假设柜台的顾客并不是严格的每一分钟过来一位,而是按照速率为 $1$ 的泊松过程过来。同样,每一位过来的顾客随机抽一张卡。我们同样考虑当所有顾客抽的卡放在一起集齐全套的时间 $T$。注意这里 $T\in \bb R$ 是一个实数。

回忆我们在上一讲讨论过的泊松过程的稀疏化(thinning)。令 $X_i(t)$ 表示在时间区间 $[0, t]$ 内,通过这个泊松过程收集到的类型 $i$ 卡片的数量。那么 $\set{X_i(t)}_{t\ge 0}$ 是一个速率为 $p_i$ 的泊松过程,并且所有这些 $X_i(t), i\in [n]$ 是相互独立的。换句话说,我们看那些抽到了第 $i$ 种卡片的人的队伍,它们的人数是各自独立的泊松过程!

对于 $i \in [n]$,令 $T_i = \min \set{ t \mid X_i(t) = 1 }$ 表示第一次开出类型 $i$ 的卡片的时间。显然 $T_i$ 的分布是 $\!{Exp}(p_i)$,并且 $T=\max_{i\in [n]} T_i$。

于是,由于独立性,我们可以计算
\[
\E{T} = \int_0^\infty \Pr{T \geq t} \d t = \int_0^\infty \left(1 - \prod_{i=1}^n (1 - e^{-p_i t})\right) \d t.
\]  

\subsection{泊松抽卡法和标准抽卡法的比较}

上面的计算可以看出,由于泊松抽卡法在稀疏化后有非常神奇的独立性质,我们可以方便的计算收集完全套的时间。直观上来说,由于是速率为 $1$ 的泊松过程,平均一分钟抽一张卡,所以,从平均的意义上来看,$\E{T}$ 很有可能和 $\E{N}$,即固定一分钟抽一张卡的平均集齐时间是很接近的。接下来,我们通过\textbf{耦合(\href{https://en.wikipedia.org/wiki/Coupling_(probability)}{\underline{coupling}})} 的方法说明事实上 $\E{T} = \E{N}$。

想象现在有两个柜台,壹号柜台是固定一分钟来一个顾客抽卡,贰号柜台是按照速率为 $1$ 的泊松过程到来顾客抽卡。我们想象,两个柜台的各自第 $i$ 位顾客总是抽到同样的卡片(这种定义联合分布的思想实验便叫做耦合)。显然,假设壹号柜台上第 $N$ 位顾客抽完卡后集齐了一套( $N$ 是随机变量),那么在贰号柜台上,也是第 $N$ 位顾客抽完卡后集齐一套。如果我们用 $\tau_i$ 表示贰号柜台第 $i-1$ 位顾客和第 $i$ 位顾客到达的间隔时间,那么 $T$ 和 $\sum_{i=1}^N \tau_i$ 有同样的分布。于是,
\[
    \E{T} = \E{\sum_{i=1}^N \tau_i}.
\]

由于 $\tau_i\sim \!{Exp}(1)$,所以 $\E{\tau_i} = 1$。在上面的式子里,如果 $N = n$ 是一个常数, 那么就有期望的线性性 $\E{\sum_{i=1}^N \tau_i} = \sum_{i=1}^n \E{\tau_i} = n$ 成立。但是,我们这里 $N$ 是一个随机变量,在最一般的情况下,期望和求和是不一定可以交换的。但在我们这儿,$N$ 和 $\set{\tau_i}$ 是独立的,\href{https://en.wikipedia.org/wiki/Wald\%27s_equation}{\underline{Wald's Equation}} 可以保证这儿交换是成立的。对于最一般的 Wald's equation,我们要学习了鞅相关的知识后才能够比较方便的证明。但在现在这个特殊情况,我们可以直接使用定义证明。

\begin{proposition}
在我们上述例子里
\[
\E{\sum_{i=1}^N \tau_i} = \E{N}.
\]

\end{proposition}

\begin{proof}
\[
    \E{\sum_{i=1}^N \tau_i} = \E{\sum_{i=1}^{\infty} \tau_i\cdot \bb I_{[i\le N]}} \overset{\mbox{(Fubini)}}{=} \sum_{i=1}^\infty \E{\tau_i\cdot \bb I_{[i\le N]}}.
\]

由于 $\tau_i$ 和 $[i\le N]$ 独立,所以 $\E{\tau_i\cdot \bb I_{[i\le N]}} = \E{\tau_i}\cdot \Pr{i\le N}$。于是
\[
    \E{\sum_{i=1}^N \tau_i} = \sum_{i=1}^\infty \Pr{N\ge i} = \E{N}.
\]

\end{proof}

这便说明了,对于非均匀的奖券收集问题,平均集齐一套的时间
\[
    \E{N} = \E{T} = \int_0^\infty \left(1 - \prod_{i=1}^n (1 - e^{-p_i t})\right) \d t.
\]

上述式子看起来比较吓人,我们接着进行一个合理性检查,即对每一个 $p_i=\frac{1}{n}$ 的时候计算一下这个积分。于是乎,
\[
\begin{aligned}
\E{N} 
&=\int_0^\infty 1-\prod_{i=1}^n \tp{1-e^{-\frac{t}{n}}} \d t\\
&\overset{(x=e^{-\frac{t}{n}})}{=} -n\int_0^1 1-(1-x)^n \d \log x\\
&=-n \int_0^1 \frac{1}{x} - \frac{(1-x)^n}{x} \d x\\
&=-n\int_0^1 \sum_{k=1}^n \frac{(1-x)^{k-1}}{x} - \frac{(1-x)^k}{x} \d x\\
&\overset{\mbox{(Fubini)}}{=}-n\sum_{k=1}^n\int_0^1 (1-x)^{k-1}\d x\\
&=n\sum_{k=1}^n \frac{1}{k} = n H_n.
\end{aligned}
\]  
神奇!

\section{泊松近似}

假设将 $m$ 个相同的球随机投放到 $n$ 个箱子中的随机试验。这个实验叫做球与箱子(\href{https://en.wikipedia.org/wiki/Balls_into_bins_problem}{\underline{Balls-into-Bins}})模型。对任意 $i \in [n]$,令 $X_i$ 表示第 $i$ 个箱子中的球的数量。如果投放是均匀随机的,那么我们有 $X_i \sim \!{Binom}\tp{m, \frac{1}{n}}$ 且 $\mathbb{E}[X_i] = \frac{m}{n}$。

这个模型可以用来建模很多概率问题,比如我们刚刚讨论过的奖券收集问题,以及生日悖论(\href{https://en.wikipedia.org/wiki/Birthday_problem}{\underline{birthday paradox}})等等。在计算机科学中,它也很自然的被用来建模随机映射的哈希表。为了理解哈希表中的冲突,我们自然会关注 $\max_{i \in [n]} X_i$ 的值,这个被称为最大负载(maxload)。然而,最大负载 $\max_{i \in [n]} X_i$ 并不是一个特别容易计算的量,原因在于 $X_i$ 之间不是相互独立的。然而,我们可以使用泊松分布来近似计算它。我们将要发展一个研究 Balls-into-Bins 问题的很一般化的工具。

\begin{theorem}
设 $\forall i\in [n], Y_i\sim\!{Pois}(\lambda)$ 是一组独立的泊松分布,其中 $\lambda>0$ 为任意固定常数。那么,在 $\sum_{i=1}^n Y_i = m$ 的条件下,$(Y_1,\dots,Y_n)$ 和 $(X_1,\dots X_n)$ 具有相同的分布。
\end{theorem}

换句话说,对于任何 $a_1,\dots,a_n$,
\[
    \Pr{(Y_1,\dots,Y_n) = (a_1,\dots,a_n)\mid \sum_{i=1}^n Y_i=m} = \Pr{(X_1,\dots,X_n) = (a_1,\dots,a_n)}.
\]

\begin{proof}
对于任意给定的 $(a_1,\dots,a_n)\in \bb N^n$ 满足 $\sum_{i=1}^n a_i=m$,我们有
\[
\Pr{(X_1, \dots, X_n) = (a_1, \dots, a_n)} = \frac{1}{n^m} \cdot \frac{m!}{a_1! a_2! \cdots a_n!}. 
\]  

另一方面,
\[
\begin{aligned}
    &\Pr{(Y_1,\dots,Y_n)=(a_1,\dots,a_n)\mid \sum_{i=1}^n Y_i=m}\\
    &\quad=\frac{\Pr{(Y_1,\dots,Y_n)=(a_1,\dots,a_n)}}{\Pr{\sum_{i=1}^n Y_i=m}}\\
    &\quad=\frac{\prod_{i=1}^n \Pr{Y_i=a_i}}{\Pr{\sum_{i=1}^n Y_i=m}}\\
    &\quad=\frac{\prod_{i=1}^n e^{-\lambda} \frac{\lambda^{a_i}}{a_i!}}{e^{-\lambda n}\frac{(\lambda n)^m}{m!}}=\frac{1}{n^m}\cdot \frac{m!}{a_1!a_2!\cdots a_n!}.
\end{aligned}
\]

\end{proof}

上面这个等分布的结论说明,当我们想计算 $(X_1,\dots,X_n)$ 的某个性质的时候,我们可以转而计算独立的 $(Y_1,\dots,Y_n)$ 的性质。当然了,我们等分布的结论需要 $\sum_i Y_i=m$ 这个条件,因此处理的方法便是把所有不满足这个条件的贡献全部扔掉。

\begin{corollary}
设 $f: \mathbb{N}^n \to \mathbb{N}$ 是可测函数,且 $Y_1, Y_2, \dots, Y_n$ 是独立的泊松随机变量,其速率为 $\lambda = \frac{m}{n}$。则有:
\[
\E{f(X_1, X_2, \dots, X_n)} \leq e\sqrt{m} \cdot \E{f(Y_1, Y_2, \dots, Y_n)}.
\]  

\end{corollary}

我们有时候把这个不等式称作泊松近似公式。

\begin{proof}
根据全期望公式,我们有:
\[
\E{f(Y_1, \dots, Y_n)} = \sum_{k=0}^\infty \E{f(Y_1, \dots, Y_n) \mid \sum_{i=1}^n Y_i = k} \cdot \Pr{\sum_{i=1}^n Y_i = k}.
\]  

由于 $f$ 是非负函数,我们扔掉所有 $k\ne m$ 的项,可以得到
\[
\E{f(Y_1, \dots, Y_n)}\ge \E{f(Y_1,\dots,Y_n)\mid \sum_{i=1}^n Y_i=m}\cdot \Pr{\sum_{i=1}^n Y_i=m}.
\] 

我们知道,假设 $Y_i\sim\!{Pois}(\lambda)$,则 $\sum_{i=1}^n Y_i\sim\!{Pois}(\lambda n)$,并且上述等式对于任意 $\lambda$ 均成立。我们希望 $\Pr{\sum_{i=1}^n Y_i=m}$ 尽量的大,根据我们计算过的泊松分布的最大似然原理,我们取 $\lambda=\frac{m}{n}$,于是根据 Stirling 公式(需要对常数进行仔细的讨论),有
\[
\Pr{\sum_{i=1}^n Y_i=m} = e^{-m}\frac{m^m}{m!} >\frac{1}{e\sqrt{m}}.
\]

所以
\[
\E{f(Y_1, \dots, Y_n)}\ge \frac{1}{e\sqrt{m}}\E{f(Y_1,\dots,Y_n)\mid \sum_{i=1}^n Y_i=m} = \frac{1}{e\sqrt{m}}\E{f(X_1,\dots,X_n)}.
\]

\end{proof}

\subsection{最大负载}

我们现在研究在 $m=n$ 时候的最大负载问题。这个问题若干年前在水源上\href{https://shuiyuan.sjtu.edu.cn/t/topic/103060}{\underline{有人}}问过。我们将证明, $m = n$ 时,最大负载 $X = \max_{i \in [n]} X_i$ 以 $1-o(1)$ 的概率,满足
\[
X = \Theta\left(\frac{\log n}{\log \log n}\right).
\]  

首先证明上界,即存在常数 $c_1>0$,使得 $\Pr{X \geq c_1 \frac{\log n}{\log \log n}} = o(1)$。令 $k = \frac{c_1 \log n}{\log \log n}$。通过 union-bound,我们有:  
\[
\Pr{X \geq k} = \Pr{\exists i \in [n], X_i \geq k} \leq \sum_{i=1}^n \Pr{X_i \geq k} = n\cdot\Pr{X_1\ge k}  .
\]

再次使用 union-bound,可以得到
\[
\Pr{X\ge k}\le n\cdot \binom{n}{k}n^{-k} \le n\tp{\frac{e}{k}}^k.
\]

注意到
\[
    k\log k  = \frac{c_1\log n}{\log\log n}\tp{\log\log n-\log\log\log n+\log c_1}
\]

取 $c_1=6$,我们有
\[
    \log n+k-k\log k<-\log n.
\]

于是,$\Pr{X\ge k}\le n\tp{\frac{e}{k}}^k<\frac{1}{n}=o(1)$。

我们接着使用泊松近似公式证明下界,即存在常数 $c_2$,使得:  
\[
\Pr{X \leq \frac{c_2\log n}{\log \log n}} = o(1).
\]  

设 $h = \frac{c_2\log n}{\log \log n}$。我们定义函数 $f(X_1,\dots,X_n)\defeq \bb I_{[X\le h]} = \bb I_{[\max_i X_i \le h]}$。于是,根据泊松近似公式
\[
  \begin{aligned}
    \Pr{X\le h}
    &=\E{f(X_1,\dots,X_n)}\\
    &\le e\sqrt{n}\E{f(Y_1,\dots,Y_n)}\\
    &=e\sqrt{n}\cdot \Pr{\max_{i\in [n]} Y_i\le h}.
  \end{aligned}
\]

根据 $Y_i$ 的定义,我们有
\[
\begin{aligned}
\Pr{\max_{i\in [n]} Y_i\le h} 
&= \tp{\Pr{Y_1\le h}}^n = (1-\Pr{Y_1>h})^n\\
&\le \tp{1-\Pr{Y_1=h+1}}^n = \tp{1-\frac{1}{(h+1)!e}}^n\le e^{-\frac{n}{e(h+1)!}}.
\end{aligned}
\]

注意到
\[
\begin{aligned}
    \log (h+1)!
    &=\sum_{i=1}^{h+1}\log i<\int_1^{h+2}\log x \d x\\
    &=(h+2)\log (h+2) -h-1\leq (h+2)\log h-h+3\\
    &=\frac{c_2\log n +2\log\log n}{\log\log n}\tp{\log\log n-\log\log\log n+\log c_2} -\frac{c_2\log n}{\log\log n}+3\\
    &\leq c_2\log n-\log\log n-2.
\end{aligned}
\]

设 $c_2=1$, 我们有 $\log (h+1)!\leq \log n-\log\log n-2$。因此
\[
    e(h+1)!\leq \frac{n}{e\log n}.
\]

所以
\[
    \Pr{\max_{i\in[n]} Y_i \le h}\leq e^{-\frac{n}{(h+1)!e}}\leq  e^{-e\log n}=n^{-e}.
\]

\newpage