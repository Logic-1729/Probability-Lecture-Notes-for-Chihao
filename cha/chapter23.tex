\chapter{高维概率,协方差矩阵,高斯分布}

\section{高维概率}

我们之前讨论过定义在同一个概率空间 $(\Omega,\@F,\bb P)$ 上的两个随机变量 $X$ 和 $Y$ 的联合分布。除了可以定义各自的期望 $\E{X},\E{Y}$ 方差 $\Var{X},\Var{Y}$ 之外,我们还能够定义协方差(\href{https://en.wikipedia.org/wiki/Covariance}{\underline{covariance}})的概念
\[
    \Cov{X,Y} \defeq \E{(X-\E{X})(Y-\E{Y})} = \E{XY} - \E{X}\E{Y}.
\]

从定义式就可以看出,协方差是方差概念的推广,因为 $\Cov{X,X} = \Var{X}$。它可以用来衡量随机变量之间的“相关度”。从定义式 $\Cov{X,Y} = \E{XY}-\E{X}\E{Y}$ 可以看出,如果 $X$ 和 $Y$ 独立,那么 $\Cov{X,Y} = 0$。但是反过来一般不成立:

\begin{example}
设 $X\sim\!{Ber}\tp{\frac{1}{2}}$; $Y = \begin{cases} \frac{1}{2} & \mbox{ if } X=0,\\ \sim\!{Ber}\tp{\frac{1}{2}} &\mbox{ if } X=1 \end{cases}$。那么
\[
\E{XY} = \E{XY\mid X=0}\Pr{X=0}+\E{XY\mid X=1}\Pr{X=1} = \frac{1}{4};
\]

并且 $\E{X}\E{Y} = \frac{1}{2}\cdot \frac{1}{2} = \frac{1}{4}$。所以 $\Cov{X,Y} = 0$。但是显然 $X$ 和 $Y$ 不独立,因为
\[
\Pr{X=0 \land Y = \frac{1}{2}} = \frac{1}{2}, \quad\Pr{X=0}\Pr{Y=\frac{1}{2}} = \frac{1}{4}.
\]

\end{example}

我们同样可以把相关概念推广到 $n$ 个随机变量。我们会用一个向量 $\*X = (X_1,\dots,X_n)^\top$ 来表示 $n$ 个随机变量 $X_1,\dots,X_n$ 所组成的向量,并把它看成 $\omega\in\Omega\mapsto \*X(\omega)\in \bb R^n$ 中的函数。我们有时候也直接称 $\*X$ 为随机向量。我们定义它的期望为
\[
\E{\*X} = \tp{\E{X_1},\E{X_2},\dots,\E{X_n}}^\top\in \bb R^n.
\]

高维随机变量在计算机科学和数据科学中都是核心的研究对象,我们这门课不会涉及太多相关的内容,感兴趣的同学可以参考两本著名的教科书 \href{https://www.math.uci.edu/~rvershyn/papers/HDP-book/HDP-book.html}{\underline{High-Dimensional Probability}} 和 \href{https://web.math.princeton.edu/~rvan/APC550.pdf}{\underline{Probability in High Dimension}}。其中前者从统计与计算机科学的角度研究高维概率,而后者从概率论的角度研究类似的对象。

对于一个 $n$-维随机变量,我们可以定义它的协方差矩阵(\href{https://en.wikipedia.org/wiki/Covariance_matrix}{\underline{covariance matrix}})为
\[
    \Cov{\*X} \defeq \E{(\*X-\E{\*X})(\*X-\E{\*X})^\top}.
\]

换句话说,$\Cov{\*X}$ 是一个 $n\times n$ 的矩阵,其第 $i$ 行 $j$ 列的元素是 $\Cov{X_i,X_j}$。协方差矩阵有一些基本的性质:
\begin{enumerate}
    \item $\Cov{\*X}$ 总是半正定的,这是因为对于任何 $\*x\in\bb R^n$,
    \[
    \*x^\top\Cov{\*X}\*x = \*x^\top\E{(\*X-\E{\*X})(\*X-\E{\*X})^\top}\*x = \E{\inner{\*x}{\*X-\E{\*X}}^2}\ge 0.
    \]
    
    \item 如果 $X_1,\dots,X_n$ 两两独立,那么 $\Cov{\*X}$ 是对角阵,并且其对角线第 $i$ 位的元素是 $\Var{X_i}$。
\end{enumerate}

注意到,我们以前介绍过的联合分布函数和联合密度函数均可以无缝推广到 $n$-维随机变量上。

\section{高斯分布}

我们开始介绍也许是最重要的高维分布,高维的高斯分布。对于任意 $i\in [n]$,我们定义 $\xi_i$ 为一个独立的 $\+N(0,1)$ 随机变量,$\xi = (\xi_1,\dots,\xi_n)$。我们把它的分布记作 $\+N(0,\!{Id}_n)$,其中 $\!{Id}_n$ 是 $n$-维的单位矩阵,它是 $\xi$ 的协方差矩阵。

我们在 $\xi$ 的基础上定义一般的高维高斯向量。我们说一个向量 $\*X$ 是\emph{高维高斯(\href{https://en.wikipedia.org/wiki/Multivariate_normal_distribution}{\underline{multi-dimensional Gaussian random variable}})},如果他可以写成 $\*X = A\xi+\mu$ 的形式,其中 $A\in \bb R^{n\times n}$,$\mu\in \bb R^n$。

我们可以计算一下 $\*X$ 的期望和协方差。首先
\[
    \E{\*X} = \E{A\xi+\mu} = A\E{\xi} + \mu = \mu.
\]

其次
\[
    \Cov{\*X} = \E{\tp{\*X-\E{X}}\tp{\*X-\E{\*X}}^\top} = \E{\tp{A\xi}\tp{A\xi}^\top} = A\E{\xi\xi^\top}A^\top = AA^\top.
\]

如果我们定义 $\Sigma = AA^\top$,那么 $\*X$ 就是一个期望为 $\mu$,协方差矩阵为 $\Sigma$ 的随机向量。我们把它的分布记为 $\+N\tp{\mu,\Sigma}$。可以看到,一个高维的高斯向量,其分布由期望和协方差矩阵唯一确定(这对于一般的随机向量显然是不对的)。

我们接下来推导 $\*X\sim\+N\tp{\mu,\Sigma}$ 的联合密度函数。我们知道,对于一个 $\+N(0,1)$ 的标准高斯随机变量,其概率密度函数为 $\phi(x) = \frac{1}{\sqrt{2\pi}}e^{-\frac{x^2}{2}}$。于是,对于 $\xi\sim\+N(0, \!{Id}_n)$,由于其各个维度均是独立的 $\+N(0,1)$ 随机变量,其概率密度函数为 $\phi_\xi(x_1,\dots,x_n) = (2\pi)^{-\frac{n}{2}}\exp\tp{-\frac{1}{2}\sum_{i=1}^n x_i^2}$。对于一般的 $\*X\sim\+N\tp{\mu,\Sigma}$,我们先贷款下面这个结论(将在下次课证明):

\begin{lemma}
设 $F$ 和 $G$ 是两个分布函数。如果对于任何一个定义在紧集上的光滑函数 $h$,都有 $\int_\Omega h \d F = \int_\Omega  h\d G$,那么在那些 $F(x)$ 连续的点 $x$ 上,$F(x) = G(x)$。
\end{lemma}

在上述引理的加持下,我们计算对于一个定义在紧集上的光滑函数 $h$ 的期望 $\E{h(\*X)}$。设 $\*X$ 的密度函数是 $\phi_{\*X}$,那么由 LOTUS:
\[
    \E{h(\*X)} = \int_{\bb R^n} h(\*x) \phi_\*X(\*x) \d\*x
\]

我们做换元 $\*x = A\*y + \mu$,并且注意到这个线性变换的 Jacobian 就是 $A$,于是
\[
    \E{h(\*X)} = \int_{\bb R^n} h\tp{A\*y+\mu} \phi_\*X(A\*y+\mu) \abs{\det A}\d\*y
\]

从另外一方面来说,如果我们按照 $\*y\sim\+N\tp{0,\!{Id}_n}$ 来积分函数 $h(A\*y+\mu)$,可以同样得到 $\E{h(\*X)}$:
\[
    \E{h(\*X)} = \int_{\bb R^n} h\tp{A\*y+\mu}\phi_\xi(\*y)\d \*y.
\]

比较上面的系数,我们可以得到 $\phi_\*X(A\*y+\mu)\abs{\det A} = \phi_\xi(\*y)$,或者等价的
\[
    \phi_\*X(\*x) = \abs{\det A}^{-1}\phi_\xi\tp{A^{-1}(\*x-\mu)} = \frac{1}{(2\pi)^{\frac{n}{2}}\tp{\det \Sigma}^{\frac{1}{2}}}\exp\tp{-\frac{1}{2}(\*x-\mu)^\top\Sigma^{-1}(\*x-\mu)}.
\]

同样,我们可以看出来,如果 $\Sigma$ 是对角阵,那么 $X_1,\dots,X_n$ 是相互独立的。这是高斯向量特有的性质。

\subsection{高斯分布的和}

我们现在证明,对于两个\textbf{独立}的高斯分布,$X_1\sim \+N(\mu_1,\sigma_1^2)$,$X_2\sim\+N(\mu_2,\sigma_2^2)$,它们的和 $X_1+X_2\sim\+N(\mu_1+\mu_2,\sigma_1^2+\sigma_2^2)$。设 $Z=X_1+X_2$,我们直接计算 $Z$ 的概率密度函数 $f_Z$。我们之前计算过两个随机变量的和的概率密度公式:
\[
\begin{aligned}
    f_Z(z) 
    &= \int_{-\infty}^{\infty}f_X(x)f_Y(y-x)\d x \\
    &=\frac{1}{2\pi\cdot\sigma_1\sigma_2}\int_{-\infty}^\infty e^{-\frac{(x-\mu_1)^2}{2\sigma_1^2}-\frac{(z-x-\mu_2)^2}{2\sigma_2^2}}\d x\\
    &=\frac{1}{2\pi\sigma_1\sigma_2}e^{-\frac{\tp{z-(\mu_1+\mu_2)}^2}{2(\sigma_1^2+\sigma_2^2)}}\int_{-\infty}^\infty e^{-\frac{\sigma_1^2+\sigma_2^2}{2\sigma_1^2\sigma_2^2}\tp{x-\frac{\sigma_2^2\mu_1+\sigma_1^2(z-\mu_2)}{\sigma_1^2+\sigma_2^2}}^2}\\
    &=\frac{1}{\sqrt{2\pi}\cdot\sqrt{\sigma_1^2+\sigma_2^2}}\exp\tp{-\frac{\tp{z-(\mu_1+\mu_2)}^2}{2(\sigma_1^2+\sigma_2^2)}}.
\end{aligned}
\]

这便是 $\+N(\mu_1+\mu_2,\sigma_1^2+\sigma_2^2)$ 的概率密度函数。上述结论可以推广到任意个 $n$ 个\textbf{相互独立}的高斯分布之和,即如果 $X_1,\dots,X_n$ 是相互独立且 $X_i\sim\+N(\mu_i,\sigma_i^2)$,那么
\[
    \sum_{i\in [n]} X_i\sim\+N\tp{\sum_{i\in [n]}\mu_i,\sum_{i\in [n]}\sigma_i^2}.
\]

注意到,上面要求的相互独立是必须的。如果两个高斯随机变量变量 $X_1$,$X_2$ 不独立,那么它们的和不一定是高斯随机变量(你能想到反例吗?)。事实上,如果我们要求 $X_1$ 和 $X_2$ 的任意线性组合都是一个高斯变量,这等价于要求 $(X_1,X_2)$ 是一个二维高斯向量。我们会在未来证明这给出了高维高斯向量的另一个等价定义。

\section{最大割的近似算法}

我们小讲一个高维向量在算法设计里面应用。我们之前研究过在一个图上求“最小割”的问题。我们今天来研究它的姊妹问题 - “最大割”。给定一个无向图 $G=(V,E)$,我们想把顶点集 $V$ 分成两部分 $S$ 和 $V\setminus S$,满足 $S$ 和 $V\setminus S$ 之间的边尽量多。在这里 $(S,V\setminus S)$ 就被称为图上的一个割。它的大小定义为 $\abs{\set{\set{u,v}\in E\cmid u\in S, v\in V\setminus S}}$。

和最小割问题不一样,最大割问题是 NP-hard 的,也就是说,如果 $\mathbf{NP}\ne \mathbf{P}$,那么最大割问题不存在多项式时间的算法能找到最优解。因此,我们期待在多项式时间内找到近似解。严格来说,我们说一个算法是 $\alpha$-近似( $\alpha\in[0,1]$ )的,当且仅当给定一个图作为输入之后,如果其最大割的值本身是 $\!{OPT}$,算法可以输出一个割,其大小至少为 $\alpha\cdot\!{OPT}$。如果 $\alpha=1$,那这就是一个最优算法。我们希望算法能够保证的 $\alpha$ 越大越好。

这是一个经典的组合优化问题,我们将介绍使用半正定规划得到的一个近似算法,这个算法大家猜想是最优的(in terms of $\alpha$ )。

\subsection{半正定规划(Positive Semi-Definite Programming, SDP)}

我们先简单介绍半正定规划,它是线性规划的推广。我们在算法或者优化课中学过线性规划:

\[
\text{max } 2x - 3y\\
\begin{aligned}
\text{s.t. } 
x + y &\leq 2, \\
3x - y &\leq 1, \\
x,y &\geq 0. 
\end{aligned}
\]

我们可以将其改写为等价的矩阵形式:

\[
\text{max } 
\begin{bmatrix}
2 & 0 \\ 
0 & -3
\end{bmatrix}
\bullet
\begin{bmatrix}
x & 0 \\ 
0 & y
\end{bmatrix}\\
\begin{aligned}
\text{s.t.}
\begin{bmatrix}
1 & 0 \\ 
0 & 1
\end{bmatrix}
\bullet
\begin{bmatrix}
x & 0 \\ 
0 & y
\end{bmatrix}
\leq 2,\\
\begin{bmatrix}
3 & 0 \\ 
0 & -1
\end{bmatrix}
\bullet
\begin{bmatrix}
x & 0 \\ 
0 & y
\end{bmatrix}
\leq 1,\\
\begin{bmatrix}
x & 0 \\ 
0 & y
\end{bmatrix}
\succeq 0.
\end{aligned}
\]

在这里,对于两个 $n \times n$ 矩阵 $A = (a_{i,j})_{1 \leq i,j \leq n}$ 和 $B = (b_{i,j})_{1 \leq i,j \leq n}$,它们 Frobenius 积定义为:

\[
A \bullet B = \sum_{1 \leq i,j \leq n} a_{i,j} \cdot b_{i,j}.
\]

在上述例子中,每个矩阵都为对角矩阵,并附加了一个正半定约束 $X \succeq 0$。

\subsubsection{SDP 的一般形式}

如上所示,标准形式的线性规划可以写成矩阵形式,其中线性规划的变量 $\set{x_i}_{i \in n}$ 被收集在一个对角矩阵 $X = \text{diag}(x_1, \dots, x_n)$ 中。正半定规划将对角矩阵 $X$ 推广到任意对称矩阵:

\[
\text{max } C \bullet X \\
\text{s.t. } A_k \bullet X \leq b_k, \, \forall k \in [m], \\
X \succeq 0.
\]

其中,$C = (c(i, j))_{1 \leq i,j \leq n}$,$X = (x(i, j))_{1 \leq i,j \leq n}$,$A_k = (a_k(i, j))_{1 \leq i,j \leq n}$ 是 $n \times n$ 的矩阵,$k \in [m]$。

我们知道一个半正定矩阵 $X$ 可以写成 $X= U^\top U$,其中 $U = \left[\*u_1,\dots,\*u_n\right]$ 满足 $\*u_i\in \bb R^n$ 是一个向量。于是 $x(i,j) = \*u_i^\top \*u_j$。我们可以使用 $\*u_i$ 们重写上面的半正定规划,从而得到
\[
\text{max } \sum_{1 \leq i,j \le n} c(i, j) \cdot \*u_i^\top \*u_j, \\
\text{s.t. } \sum_{1 \leq i,j \le n} a_k(i, j) \cdot \*u_i^\top \*u_j \leq b_k, \, \forall k \in [m], \\
\*u_i \in \mathbb{R}^n, \, i \in [n].
\]

这被称为向量规划,与半正定规划是等价的。与线性规划类似,只要提供一个高效的 \href{https://en.wikipedia.org/wiki/Separation_oracle}{\underline{seperation oracle}},就可以使用椭球法(\href{https://en.wikipedia.org/wiki/Ellipsoid_method}{\underline{ellipsoid method}})在多项式时间内(近似)求解。一个  seperation oracle 指的是如下一个算法:给定一个点 $\*x\in\bb R^n$:

\begin{enumerate}
\item 如果 $\*x$ 是可行解,算法输出 YES;
    \item 如果 $\*x$ 不是可行解,算法输出一个它所不满足的约束。
\end{enumerate}

我们并不想详细介绍 SDP,我们只是把上述结果当成黑盒使用。

\subsection{Goemans-Williamson 的舍入算法}

实际上,我们可以把最大割问题自然的建模成一个二次规划(\href{https://en.wikipedia.org/wiki/Quadratic_programming}{\underline{quadratic programming}}):

\[
\text{max } \frac{1}{2} \sum_{e = \{u, v\} \in E} (1 - x_u x_v)\\
\text{s.t. } x_u \in \{-1, 1\}, \, \forall u \in V.
\]

对于这样一个规划,由于其与最大割问题是等价的,我们知道它也是 NP-hard 的。近似算法设计的一个常用技巧是,把这个规划放松(relax)成一个取值范围更大,也因此更容易解的优化问题。然后从这个容易解的优化问题的最优解得到原规划的最优解。后面一步通常被称为舍入(rounding)。在我们这个问题里,如果将每个 $x_u$ 视为一个一维向量 $x_u \in \mathbb{R}^1$,我们可以将其放松为一个 $n$ 维向量 $\*w_u \in \mathbb{R}^n$。于是,我们可以得到如下向量规划:
\[
\text{max } \frac{1}{2} \sum_{e = \{u, v\} \in E} \left(1 - \*w_u^\top \*w_v\right)\\
\text{s.t. } \*w_u \in \mathbb{R}^n, \, \forall u \in V; \quad \*w_u^\top \*w_u = 1, \, \forall u \in V。
\]

使用之前提到的求解半正定规划的黑盒,我们可以高效的求解上述规划,并得到向量规划的最优解 $\{\*w_u^*\}_{u \in V}$。注意到,每一个 $\*w^*_u$ 均是 $\bb R^n$ 中的一个向量。直观上,这些向量包含了一个好的割的信息:如果 $\*w_u^\top\*w_v$ 比较小,说明 $u$ 和 $v$ 更应该被割开。那我们如何从它舍入成一个割,并严格的证明其近似比呢?

Goemans-Williamson 舍入算法通过随机采样一个穿过原点的超平面,将向量分为两部分:超平面一侧的向量,对应于 $S$,和另一侧的向量,对应于 $V\setminus S$。于是,角度较大的向量对更可能被分开,这也与我们前面说的直观是一致的。

\subsubsection{如何随机采样超平面?}

这并不是一个简单的问题,因为我们首先要定义随机超平面的概率空间,这稍微有一点麻烦。我们便如同大部分计算机科学中做的一样,稍微不那么严格一点,并把正确性的验证付诸于直观(但我并不支持这样做,大家可以参看前文提到的 High-Dimensional Probability 教材来寻找严格的处理)。我们做的具体方法是从 $n-1$ 维单位球( $S^{n-1} = \set{x \in \mathbb{R}^n \cmid \|x\| = 1}$ )中均匀地采样一个点作为超平面的法向量。$S_{n-1}$ 上的均匀测度是一个 \href{https://en.wikipedia.org/wiki/Haar_measure}{\underline{Haar 测度}},它是存在的,我们暂时接受这个设定。我们现在来说明如何从这个测度中进行采样。事实上,我们只需要取一个 $\*x\sim\+N(0,\!{Id}_n)$,然后输出 $\frac{\*x}{\norm{\*x}}$ 即可。这是因为我们知道
\[
    \phi_\xi(\*x) \propto \exp\tp{-\frac{1}{2}\|\*x\|^2},
\]

仅仅依赖于 $\*x$ 的模长 $\|\*x\|$。因此归一化后,$\frac{\*x}{\|\*x\|}$ 在 $S^{n-1}$ 上均匀分布。

\begin{example}[Goemans-Williamson 舍入算法]

\begin{enumerate}
    \item 计算 $\{\*w_u^*\}_{u \in V}$。
    \item 随机选择一个向量 $\*x = (x_1, \dots, x_n) \in S^{n-1}$。
    \item 定义集合 $S = \set{u \in V \cmid \*x^\top \*w_u^* \geq 0}$。
\end{enumerate}
\end{example}

\begin{theorem}
Goemans-Williamson 舍入算法是最大割问题的一个随机 $\alpha^*$-近似算法,其中 $\alpha^* > 0.878$。
\end{theorem}

记 $w_u^*$ 和 $w_v^*$ 之间的夹角为 $\theta_{u,v}$,即 $\theta_{u,v} = \arccos(\inner{w_u^*}{ w_v^*})$。由于分隔超平面是均匀选取的,对于任意边 $e = \{u, v\} \in E$,顶点 $u$ 和 $v$ 被分隔(位于超平面两侧)的概率为:

\[
\Pr{u\mbox{ and }v\mbox{ are seperated}} = \frac{\theta_{u,v}}{\pi}.
\]

我们用随机变量 $X$ 表示割的大小,则有:

\[
\E{X} = \sum_{\{u, v\} \in E} \Pr{u \mbox{ and } v \mbox{ are seperated}} = \sum_{\{u, v\} \in E} \frac{\arccos(\*w_u^* \cdot \*w_v^*)}{\pi}。
\]

设 $\alpha^* = \min_{-1 \leq x \leq 1} \frac{2 \arccos x}{\pi (1 - x)} > 0.878$。则有
\[
\E{X} \geq \alpha^* \cdot \!{OPT}\mbox{-}\!{VP}\geq \alpha^* \cdot \!{OPT}.
\]

其中 $\!{OPT}\mbox{-}\!{VP}$ 是向量规划的最优解,而 $\!{OPT}$ 是最大割的最优解。

Goemans-Williamson 舍入算法看似是一个对于这个问题很特殊的算法,也看似有很多改进的空间。同样0.878也看似是一个无厘头的数。但在\href{https://en.wikipedia.org/wiki/Unique_games_conjecture}{\underline{某一些计算复杂性假设}}下,这个近似比是最优的。
