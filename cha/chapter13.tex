\chapter{乘积概率空间,富比尼-托内利定理}

在我们上周的作业里,我们让大家举反例说明如下命题是$\textbf{不对}$的:如果每一个 $X_i$ 均可积,那么
\[
\mathbb{E}\left[\sum_{i=1}^\infty X_i\right] = \sum_{i=1}^\infty \mathbb{E}[X_i].
\]

我们知道,无论是期望还是求和,都可以看成在某个测度下的勒贝格积分。因此,我们今天想来回答,在什么条件下,积分是可以交换的,即
\[
\int_A \left( \int_B f  dP \right) dQ = \int_B \left( \int_A f  dQ \right) dP
\]

成立。我们首先来定义乘积测度空间。

\section{乘积概率空间}

假设我们有两个概率空间 $(X, \mathscr{X}, \mu)$ 和 $(Y, \mathscr{Y}, \nu)$。我们可以想象:

\begin{itemize}
\item 第一个概率空间是掷一个骰子得到 $\omega_1 \in X$,第二个概率空间是掷另一个骰子,得到 $\omega_2 \in Y$;或者
    \item 第一个概率空间是在 $[0, 1]$ 上均匀取一个数 $a$,第二个概率空间是在 $[0, 2]$ 上均匀取一个数 $b$。
\end{itemize}

我们现在可以考虑一个概率空间,称为乘积概率空间,它的样本集是 $X \times Y$,用来表示独立的做两次实验。在我们上面的两个例子里,分别对应了:
\begin{itemize}
\item 同时投两枚骰子,得到 $(\omega_1, \omega_2)$;
    \item 在 $[0, 1] \times [0, 2]$ 上均匀取一个点 $(a, b)$。
\end{itemize}

$\phantom{aaaaa}$

所以一个自然的问题是,这个乘积概率空间里,事件集和测度应该是什么。离散概率空间比较平凡,我们考虑上面第二个例子。这个时候 $\mathscr{X} = \mathscr{B}([0, 1])$,$\mathscr{Y} = \mathscr{B}([0, 2])$。集合 $\mathscr{X} \times \mathscr{Y} = \{ A \times B : A \in \mathscr{X}, B \in \mathscr{Y} \}$ 是 $[0, 1] \times [0, 2]$ 上所有“矩形”的集合。它显然不是一个 $\sigma$-代数,因此,我们可以取乘积概率空间的事件集为 $\sigma(\mathscr{X} \times \mathscr{Y})$,并把它记做 $\mathscr{X} \otimes \mathscr{Y}$。

更一般的,对于两个测度空间 $(X, \mathscr{X}, \mu)$ 和 $(Y, \mathscr{Y}, \nu)$,我们定义它的乘积空间为 $(X \times Y, \mathscr{X} \otimes \mathscr{Y})$,其中 $\mathscr{X} \otimes \mathscr{Y} := \sigma(\mathscr{X} \times \mathscr{Y})$。我们接下来定义 $\mathscr{X} \otimes \mathscr{Y}$ 上的测度。看起来有两种比较自然的定义方式。

给定一个集合 $E \in \mathscr{X} \otimes \mathscr{Y}$,我们定义两个函数 $E_1 : x \in X \mapsto E_1(x) \subseteq Y$ 和 $E_2 : y \in Y \mapsto E_2(y) \subseteq X$,分别表示集合 $E$ 在 $X = x$ 和 $Y = y$ 上的投影,如下图所示。
\[
E_1(x) := \{ y \in Y : (x, y) \in E \}, \quad E_2(y) := \{ x \in X : (x, y) \in E \}.
\]

\begin{figure}[h]
\centering
\includegraphics[width=0.3\textwidth]{figure/Figure13.1.png}
\caption{Figure1}
\end{figure}

我们需要说明,如果 $E$ 是 $\mathscr{X} \otimes \mathscr{Y}$ 可测的,那么对于任意 $x$ 和 $y$,$E_1(x)$ 与 $E_2(y)$ 分别是 $\mathscr{Y}$ 与 $\mathscr{X}$ 可测的。

\begin{proposition}
如果 $E$ 是 $\mathscr{X} \otimes \mathscr{Y}$ 可测的,那么对于任意 $x$ 和 $y$,$E_1(x)$ 与 $E_2(y)$ 分别是 $\mathscr{Y}$ 与 $\mathscr{X}$ 可测的。
\end{proposition}

我们把命题的证明放到本次讲义的最后,先来继续我们的讨论。

于是,我们可以定义
\[
\pi_1(E) := \int_X \nu(E_1(x))  \mu(dx); \quad \pi_2(E) := \int_Y \mu(E_2(y))  \nu(dy).
\]

换句话说,$\pi_1(E)$ 和 $\pi_2(E)$ 分别对应着在计算 $E$ 的面积的时候“横着积”和“竖着积”。设
\[
\mathscr{G} = \{ E \in \mathscr{X} \otimes \mathscr{Y} : \pi_1(E) = \pi_2(E) \}
\]

为所有 $\pi_1$ 和 $\pi_2$ 相等的可测集的集合。对于矩形 $E = A \times B \in \mathscr{X} \times \mathscr{Y}$,我们显然有
\[
\pi_1(E) = \pi_2(E) = \mu(A) \times \nu(B),
\]

即 $\mathscr{X} \times \mathscr{Y} \subseteq \mathscr{G}$。如果 $\mu$ 和 $\nu$ 均是 $\sigma$-有限的,我们可以使用单调类定理说明 $\mathscr{X} \otimes \mathscr{Y} \subseteq \mathscr{G}$(留作练习)。所以,我们可以定义测度
\[
\forall E \in \mathscr{X} \otimes \mathscr{Y}, \quad (\mu \otimes \nu)(E) := \pi_1(E) = \pi_2(E).
\]

于是,$(X \times Y, \mathscr{X} \otimes \mathscr{Y}, \mu \otimes \nu)$ 便是我们构造出来的乘积空间。

我们考虑一个特殊的例子。设 $X = Y = \mathbb{R}$,$\mathscr{X} = \mathscr{Y} = \mathscr{B}(\mathbb{R})$,$\mu = \nu = \lambda$。那么 $(X, \mathscr{X}, \mu)$ 和 $(Y, \mathscr{Y}, \nu)$ 均为一维的 $\mathbb{R}$ 上的勒贝格测度空间。按照我们的定义,它的乘积空间是 $(\mathbb{R}^2, \mathscr{B}(\mathbb{R}) \otimes \mathscr{B}(\mathbb{R}), \lambda \otimes \lambda)$。同样根据定义,我们知道 $\mathscr{B}(\mathbb{R}) \otimes \mathscr{B}(\mathbb{R})$ 就是 $\mathscr{B}(\mathbb{R}^2)$,并且根据 Carathéodory 定理,$\lambda \otimes \lambda$ 就是 $\mathbb{R}^2$ 上的勒贝格测度。

我们的定义可以推广到任意有限个概率空间的乘积。对于无穷多个概率空间的乘积,情况比较复杂,超出了这门课的范畴,可以参见 \href{https://en.wikipedia.org/wiki/Kolmogorov_extension_theorem}{\underline{Kolmogorov extension theorem}}。

\section{富比尼-托内利定理(\href{https://en.wikipedia.org/wiki/Fubini\%27s_theorem}{\underline{Fubini-Tonelli's Theorem}})}

富比尼-托内利定理说的是,如果一个定义在乘积空间 $(X \times Y, \mathscr{X} \otimes \mathscr{Y}, \mu \otimes \nu)$ 上的可测函数 $f$ 非负,或者可积,那么有等式
\[
\int_{X \times Y} f(x, y)  \mu \otimes \nu(dx dy) = \int_X \left( \int_Y f(x, y)  \nu(dy) \right) \mu(dx) = \int_Y \left( \int_X f(x, y)  \mu(dx) \right) \nu(dy)
\]

成立。

也就是说,积分的顺序可以任意交换。当然,这个等式成立需要先保证对于任意 $x$,函数 $f(x, \cdot) : y \in Y \mapsto f(x, y) \in \mathbb{R}$ 是可测的(对函数 $f(\cdot, y)$ 同理)。这是上一节关于 $E_1$ 和 $E_2$ 函数可测的更一般版本。我们将在最后一节证明。

值得强调的是,关于 $f$ 的可积性要求是 $|f|$ 在 $\mu \otimes \nu$ 这个乘积测度上是可积的,如果仅仅是对于任何 $y \in Y$, $f(\cdot, y)$ 在 $\mu$ 上可积,或者对于任何 $x \in X$,$f(x, \cdot)$ 在 $\nu$ 上可积,是不够的。我们本次课一开始提到的那个反例就说明了这一点。

我们简单叙述一下定理的证明,类似的套路大家现在应该很熟悉了,具体的细节请大家自己完成。我们只讨论 $f \geq 0$ 的情况,可积的情形可以转化为分别考虑 $f^+$ 与 $f^-$ 两个非负函数。

\begin{enumerate}

\item 首先,如果 $f = \mathbb{I}_E$,其中 $E$ 是某个可测集。那么此时,我们想要证明的等式即我们上一节验证过的“横着积”等于“竖着积”的测度等式。

\item 然后,我们考虑 $f$ 的下近似 $\underline{f}_n$。对于每一个 $n$,它都可以写成 $\sum_i c_i \cdot \mathbb{I}_{E_i}$ 的形式,其中 $E_i$ 是可测集。于是,使用期望的线性性,我们可以证明此时的等式。

\item 使用 MCT,我们可以对 $\underline{f}_n$ 的积分求极限,得到对 $f$ 的积分。

\end{enumerate}

在具体使用 Fubini-Tonelli 的时候,我们往往是这样做的:
\begin{itemize}
\item 如果 $f$ 非负,那么可以随意的交换积分顺序进行计算。
    \item 如果 $f$ 可正可负,我们先对 $|f|$ 计算某一个累次积分,比如 $\int_X(\int_Y |f| d\nu)d\mu$。如果其 $< \infty$,那么就说明它满足我们定理的条件。
\end{itemize}

\section{$f(x, \cdot)$, $f(\cdot, y)$ 可测的证明}

我们接下来证明在乘积测度空间 $(X \times Y, \mathscr{X} \otimes \mathscr{Y}, \mu \otimes \nu)$ 上的可测函数 $f:(x, y) \in X \times Y \mapsto \mathbb{R}$ 的限制 $f(x, \cdot) : y \in Y \mapsto f(x, y)$ 和 $f(\cdot, y) : x \in X \mapsto f(x, y)$ 分别是在 $\mathscr{Y}$ 和 $\mathscr{X}$ 上可测的。

我们证明 $f(x, \cdot)$ 的情况。考虑一个固定的 $x \in X$ 以及一个函数
\[
T_x : y \in Y \mapsto (x, y) \in X \times Y.
\]

显然,$f(x, \cdot) = f \circ T_x$。我们只需要验证 $T_x$ 是从 $(X, \mathscr{X})$ 到 $(X \times Y, \mathscr{X} \otimes \mathscr{Y})$ 的可测函数即可。因为容易验证,两个可测函数的复合函数还是一个可测函数。

为了验证 $T_x$ 是一个可测函数,我们只需要验证,对于每一个 $\mathscr{X} \otimes \mathscr{Y}$ 里的矩形 $A \times B$,$T_x^{-1}(A \times B) \in \mathscr{Y}$ 即可(我们以前证明过这件事情,还记得吗)。但
\[
T_x^{-1}(A \times B) = \begin{cases} B, & \text{if } x \in A; \\ \varnothing, & \text{if } x \notin A. \end{cases}
\]

因此 $T_x$ 是可测的。

\newpage