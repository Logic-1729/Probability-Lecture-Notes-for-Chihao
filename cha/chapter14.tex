\chapter{联合分布,联合密度函数,条件密度函数}

我们之前介绍了一个随机变量的分布函数、分布、概率质量/密度函数等,今天,我们开始介绍定义在同一个概率空间上的多个随机变量的“联合”分布。

\section{联合分布(\href{https://en.wikipedia.org/wiki/Joint_probability_distribution}{\underline{Joint Distribution}})}

我们还是固定一个概率空间 $(\Omega, \mathscr{F}, \mathbb{P})$。对于定义在上面的两个随机变量 $X, Y: \Omega \to \mathbb{R}$,我们定义它们的联合分布函数 $F: \mathbb{R}^2 \to [0, 1]$ 为
\[
\forall x, y \in \mathbb{R}, \quad F(x, y) := \mathbb{P}[X \leq x, Y \leq y].
\]

这个定义可以直接推广成任意有限个随机变量的联合分布
\[
\forall x_1, \dots, x_n \in \mathbb{R}, \quad F(x_1, \dots, x_n) := \mathbb{P}[X_1 \leq x_1, \dots, X_n \leq x_n].
\]

对于一般的 $n$,联合分布的大部分性质和 $n = 2$ 时并没有本质区别,因此,我们接下来的讨论均以 $n = 2$ 为例。除非额外说明,我们所述的性质都可以被推广到一般的有限 $n$。

\begin{proposition}[多元分布函数的性质]
\begin{enumerate}
\item $x \mapsto F(x, y)$ 以及 $y \mapsto F(x, y)$ 均是左极限存在,右连续的非降函数;
    \item $F(x, \infty) := \lim_{y \to \infty} F(x, y)$,$F(\infty, y) := \lim_{x \to \infty} F(x, y)$ 均存在;
    \item 对于每一个 $x, y$,$\lim_{x \to -\infty} F(x, y) = \lim_{y \to -\infty} F(x, y) = 0$,$\lim_{x,y \to \infty} F(x, y) = 1$;
    \item $\mathbb{P}[X = x, Y = y] = F(x, y) - F(x-, y) - F(x, y-) + F(x-, y-)$,其中 $F(x-, y) := \lim_{u \uparrow x} F(u, y)$(其余类似)。
\end{enumerate}
\end{proposition}

当我们谈论 $X$ 和 $Y$ 的联合分布函数的时候,有时候会把 $X, Y$ 作为下标,记作 $F_{XY}$。我们定义 $\textbf{边缘分布函数}$(marginal distribution function)
\[
F_X(x) := F_{XY}(x, \infty), \quad F_Y(y) := F_{XY}(\infty, y).
\]

显然, $F_X(x)$ 和 $F_Y(y)$ 就是 $X$ 和 $Y$ 对应的分布函数。我们这儿称之为“边缘”的原因是强调它们分别是一个联合分布的一部分。

\subsection{概率质量函数与概率密度函数}

对于离散随机变量 $X$ 和 $Y$,我们有$\textbf{联合质量函数}$
\[
p_{XY}(x, y) := \mathbb{P}[X = x, Y = y].
\]

显然,随机变量 $X$ 和 $Y$ 的概率质量函数 $p_X$ 和 $p_Y$ 分别满足
\[
p_X(x) = \sum_{y \in \mathsf{Im}(Y)} p(x, y), \quad p_Y(y) = \sum_{x \in \mathsf{Im}(X)} p(x, y).
\]

我们有时候把它们称为 $X$ 或者 $Y$ 对应的$\textbf{边缘概率质量函数}$(marginal probability mass function)。

类似的,假设 $X$ 和 $Y$ 是连续随机变量,如果存在一个非负的函数 $f(x, y)$,满足
\[
F(x, y) = \int_{-\infty}^y \int_{-\infty}^x f(u, v)  du  dv,
\]

则称 $f(x, y)$ 是 $X$ 和 $Y$ 的$\textbf{联合概率密度}$。我们可以使用单调类定理说明,对于任何的 $A \in \mathscr{B}(\mathbb{R}^2)$,
\[
\mathbb{P}[(X, Y) \in A] = \int_A f(x, y)  dx \otimes dy.
\]

由微积分基本定理,如果 $f$ 在 $(x, y)$ 连续,那么
\[
f(x, y) = \frac{\partial^2}{\partial x \partial y} F(x, y).
\]

我们同样可以定义$\textbf{边缘密度函数}$ $f_X$ 和 $f_Y$ 为
\[
f_X(x) := \int_{-\infty}^\infty f(x, y)  dy, \quad f_Y(y) := \int_{-\infty}^\infty f(x, y)  dx.
\]

容易验证,它们实际上分别是 $F_X$ 和 $F_Y$ 的密度函数。

值得注意的是,如果 $(X, Y)$ 具有联合密度函数,那么它们就有边缘密度函数,但反过来不一定成立。比如 $X$ 是 $(0, 1)$ 上均匀取的一个数,$Y = X$,容易验证,$(X, Y)$ 不存在联合密度函数(why?)。

如果 $X$ 和 $Y$ 有连续的联合密度函数 $f_{XY}(x, y)$,那么 $X$ 和 $Y$ 独立当且仅当 $f_{XY}(x, y) = f_X(x) f_Y(y)$。为了说明这一点,我们只需要注意到
\[
f_{XY}(x, y) = \frac{\partial^2}{\partial x \partial y} F_{XY}(x, y) = \frac{\partial^2}{\partial x \partial y}(F_X(x) F_Y(y)) = f_X(x) f_Y(y)
\]

即可。

\section{条件分布与条件密度(\href{https://en.wikipedia.org/wiki/Conditional_probability_distribution}{\underline{Conditional Distribution}})}

我们接下来讨论条件概率。我们在之前介绍概率空间的时候已经定义过条件概率了。给定两个事件 $A, B \in \mathscr{F}$,如果 $\mathbb{P}[B] \neq 0$,那么我们定义条件概率
\[
\mathbb{P}[A \mid B] = \frac{\mathbb{P}[A \cap B]}{\mathbb{P}[B]}.
\]

这个定义可以自然的给出离散的随机变量的条件期望的定义。假设 $X$ 是一个离散的随机变量,那么,对于任何可测集 $A$ 和 $x$,如果 $\mathbb{P}[X = x] > 0$,那么,我们可以无缝使用上面的定义得到
\[
\mathbb{P}[Y \in A \mid X = x] = \frac{\mathbb{P}[Y \in A \wedge X = x]}{\mathbb{P}[X = x]}.
\]

如果 $\mathbb{P}[X = x] = 0$,这个时候 $\mathbb{P}[Y \in A \mid X = x]$ 是无定义的。我们可以同时自然的定义出
条件分布函数
\[
F_{Y \mid X}(y \mid x) := \mathbb{P}[Y \leq y \mid X = x];
\]

以及得到对应的
条件质量函数
\[
p_{Y \mid X}(y \mid x) = \begin{cases} \dfrac{p_{YX}(y, x)}{p_X(x)}, & \text{if } p_X(x) > 0; \\ 0, & \text{otherwise}. \end{cases}
\]

我们可以同时给出条件期望的定义。如果 $Y$ 可积并且 $\mathbb{P}[X = x] > 0$,那么定义
\[
\mathbb{E}[Y \mid X = x] := \frac{\mathbb{E}[Y \cdot \mathbb{I}_{X=x}]}{\mathbb{P}[X = x]}.
\]

上面这些定义都是非常自然,而且我们之前在作业里也多次显式或者隐式的使用过了。但是,当 $X$ 不是离散随机变量的时候,这样的定义就会出现一些问题。比如说,假设 $X$ 和 $Y$ 是独立的从 $[0, 1]$ 中均匀得到的两个数,那么直观上,我们应该有 $\mathbb{P}[Y \leq \frac{1}{2} \mid X = \frac{1}{3}] = \frac{1}{2}$。但由于 $\mathbb{P}[X = \frac{1}{3}] = 0$,我们上述给出的条件概率定义是一个形如 $\frac{0}{0}$ 的没有意义的数。因此,我们需要对条件概率有新的定义。实际上,在概率论里面,条件概率是\href{https://en.wikipedia.org/wiki/Conditional_expectation}{\underline{条件期望}}的特殊情况,而\href{https://en.wikipedia.org/wiki/Conditional_expectation#Conditional_expectation_with_respect_to_a_sub-\%CF\%83-algebra}{\underline{最一般}}的条件期望的定义,我们现在还没有准备好。

大约在这门课的最后,我们会给出定义。今天,我们先讨论一个特殊情况,即在 $X$ 和 $Y$ 有连续的联合密度函数$ f_{XY}$的时候,定义条件期望与条件概率。

我们刚才说了,由于 $\mathbb{P}[X = x] = 0$,我们从近似的角度来考虑这个问题。根据微积分基本定理,对于一个很小的 $h > 0$,我们有
\begin{align*}
\mathbb{P}\left[ Y \le y \mid X \in [x, x + h] \right]
&= \frac{
    \int_{-\infty}^{y} \int_{x}^{x+h} f_{XY}(u, v) \, du \, dv
}{
    \int_{x}^{x+h} f_X(u) \, du
} \\
&= \frac{
    \int_{-\infty}^{y} h \cdot f_{XY}(x, v) + o(h) \, dv
}{
    (h + o(h)) f_X(x)
} \\
&= \frac{
    \int_{-\infty}^{y} f_{XY}(x, v) \, dv + h^{-1} \int_{-\infty}^{y} o(h) \, dv
}{
    f_X(x) + o(1)
}.
\end{align*}

如果我们假设 $f_{XY}$ 有一定的正则性使得 $\lim_{h \to 0} h^{-1} \int_{-\infty}^y o(h)  dv = \int_{-\infty}^y \lim_{h \to 0} h^{-1} o(h) = 0$,则我们可以对于可测的 $A$,定义
\[
\mathbb{P}[Y \in A \mid X = x] := \lim_{h \to 0} \mathbb{P}[Y \in A \mid X \in [x, x+h]].
\]

更一般的($f_{XY}$ 不一定连续),我们可以自然的定义条件分布函数
\[
F_{Y \mid X}(y \mid x) := \begin{cases} \int_{-\infty}^y \dfrac{f_{XY}(x, v)}{f_X(x)}  dv, & \text{if } f_X(x) > 0, \\ 0, & \text{if } f_X(x) = 0. \end{cases}
\]

其对应的条件密度函数为
\[
f_{Y \mid X}(y \mid x) = \begin{cases} \dfrac{f_{XY}(x, y)}{f_X(x)}, & \text{if } f_X(x) > 0, \\ 0, & \text{if } f_X(x) = 0. \end{cases}
\]

我们也定义条件期望
\[
\mathbb{E}[Y \mid X = x] := \int_{-\infty}^\infty y f_{Y \mid X}(y \mid x)  dy.
\]

条件期望是一个非常重要的概念,我们在未来会专门讨论条件期望的性质并给出对应的应用,今天,我们暂时了解这个定义即可。

我们接着验证一下,所谓全概率公式,对于具有连续联合密度的随机变量也成立。

\begin{proposition}[全概率公式]
\[
\mathbb{P}[Y \in A] = \int_{-\infty}^\infty \int_A f_{Y \mid X}(y \mid x) f_X(x)  dy  dx.
\]
\end{proposition}

我们仅需要把定义代进去,并使用 Fubini-Tonelli 交换积分顺序即可证明。注意到
\begin{align*}
\int_{-\infty}^{\infty} \int_A f_{Y|X}(y|x) f_X(x) \, dy \, dx
&= \int_{-\infty}^{\infty} \int_A \frac{f_{XY}(x, y)}{f_X(x)} \cdot f_X(x) \, dy \, dx \\
&= \int_{-\infty}^{\infty} \int_A f_{XY}(x, y) \, dy \, dx \\
&= \int_A f_Y(y) \, dy \\
&= \mathbb{P}\left[ Y \in A \right].
\end{align*}

使用类似的证明,我们可以更一般的得到,对于 $A, B \in \mathscr{F}$,
\[
\mathbb{P}[Y \in A \wedge X \in B] = \int_B \int_A f_{Y \mid X}(y \mid x) f_X(x)  dy  dx.
\]

\subsection{积分的换元}

我们现在考虑一个在计算中经常会遇到的问题,假设我们知道随机变量 $X$ 和 $Y$ 的联合密度函数 $f_{XY}$,那么对于新的随机变量 $(U, V) = g(X, Y) = (g_1(X, Y), g_2(X, Y))$,它们的联合密度函数 $f_{UV}$ 是什么?这儿 $g_1, g_2: \mathbb{R}^2 \to \mathbb{R}$ 是两个可测函数,并且我们假设它们是可微的。

对于一个可积的测试函数 $\phi: \mathbb{R}^2 \to \mathbb{R}$,我们考虑用两种方法来计算 $\mathbb{E}[\phi(U, V)]$。首先是通过 $U, V$ 的联合密度函数 $f_{UV}$:
\[
\mathbb{E}[\phi(U, V)] = \int_{-\infty}^\infty \int_{-\infty}^\infty \phi(u, v) f_{UV}(u, v)  du  dv.
\]

接着是通过 $X, Y$ 的联合密度函数 $f_{XY}$:
\[
\mathbb{E}[\phi(U, V)] = \mathbb{E}[\phi(g(X, Y))] = \int_{-\infty}^\infty \int_{-\infty}^\infty \phi(g(x, y)) f_{XY}(x, y)  dx  dy.
\]

我们再把上面第一个式子使用换元公式得到
\[
\int_{-\infty}^\infty \int_{-\infty}^\infty \phi(u, v) f_{UV}(u, v)  du  dv = \int_{-\infty}^\infty \int_{-\infty}^\infty \phi(g(x, y)) f_{UV}(g(x, y)) |\det J_g(x, y)|  dx  dy,
\]

其中 $J_g(x, y)$ 是 $g$ 在 $(x, y)$ 处的雅可比矩阵
\[
J_g(x, y) = \begin{pmatrix} \frac{\partial g_1}{\partial x} & \frac{\partial g_1}{\partial y} \\ \frac{\partial g_2}{\partial x} & \frac{\partial g_2}{\partial y} \end{pmatrix}.
\]

所以,我们可以得到如下命题:

\begin{proposition}
\[
f_{XY}(x, y) = f_{UV}(g(x, y)) |\det J_g(x, y)|.
\]
\end{proposition}

\begin{example}[极坐标的例子]
我们考虑下面的例子,假设 $X$ 和 $Y$ 是两个独立的标准正态分布随机变量,那么它们的联合密度函数为 $f_{XY}(x, y) = \frac{1}{2\pi} e^{-\frac{x^2 + y^2}{2}}$。我们可以把 $(X, Y)$ 看成 $\mathbb{R}^2$ 上的随机的点。我们考虑这些点的极坐标 $(R, \Theta)$,其中 $R = \sqrt{X^2 + Y^2}$,$\Theta = \arctan \frac{Y}{X}$。我们想知道 $(R, \Theta)$ 的联合密度函数是什么。

我们首先知道,$X = R \cos \Theta$,$Y = R \sin \Theta$。这个变换的雅可比矩阵的行列式是 $r$。因此,根据 命题14.3,我们有
\[
f_{R\Theta}(r, \theta) = f_{XY}(r \cos \theta, r \sin \theta) \cdot r = \frac{r}{2\pi} e^{-\frac{r^2}{2}}.
\]

大家会发现这个式子是与 $\theta$ 无关的,这说明关于 $\theta$ 的边缘分布是均匀分布。这件事情的一个推论是,如果我们希望从二维的单位圆上均匀的取出一个点来,我们只需独立的取两个标准高斯变量 $(X, Y)$,然后把它归一化成长度为 $1$ 的向量 $\left( \frac{X}{\sqrt{X^2+Y^2}}, \frac{Y}{\sqrt{X^2+Y^2}} \right)$ 即可。这件事情对于高维也是成立的,对于算法设计很有意义。
\end{example}

\begin{example}[随机变量的和]
假设知道 $X$ 和 $Y$ 的联合概率密度 $f_{XY}$,我们来考虑两个随机变量的和 $Z = X + Y$ 的概率密度。我们首先引入一个辅助变量 $W = Y$,于是对于 $(Z, W)$, $(X, Y) = g(Z, W)$,我们有
\[
|\det J_g(z, w)| = \begin{vmatrix} 1 & -1 \\ 0 & 1 \end{vmatrix} = 1.
\]

所以根据 命题14.3,我们有
\[
f_{ZW}(z, w) = f_{XY}(z - w, w).
\]

我们可以计算出 $Z$ 的边缘密度函数为
\[
f_Z(z) = \int_{-\infty}^\infty f_{ZW}(z, w)  dw = \int_{-\infty}^\infty f_{XY}(z - w, w)  dw = \int_{-\infty}^\infty f_Y(w) \cdot f_{X \mid Y}(z - w \mid w)  dw.
\]

特别的,如果 $X$ 和 $Y$ 独立,那么
\[
f_Z(z) = \int_{-\infty}^\infty f_X(z - w) f_Y(w)  dw.
\]
\end{example}

\newpage