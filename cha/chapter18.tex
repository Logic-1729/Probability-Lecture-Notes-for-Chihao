\chapter{随机变量收敛的模式,Borel-Cantelli 引理}

\section{随机变量的收敛(\href{https://en.wikipedia.org/wiki/Convergence_of_random_variables}{\underline{Convergence of Random Variables}})}

在概率论的研究里面,我们经常会讨论(定义在同一个概率空间上的)随机变量列 $\set{X_n}_{n\ge 1}$ 收敛到 $X$。我们之前已经遇到过所谓的“几乎必然”收敛 $X_n\overset{a.s.}{\to} X$。我们今天将介绍其它几种常见的收敛模式。这些收敛模式会出现在概率论研究的不同场合。比如说,在这门课未来学习大数定律的时候我们会接触“依概率”收敛,在未来学习随机过程的时候会遇到“依 $L^2$ ”收敛,以及我们在计算中常常用到的“依分布”收敛。我们现在先给出各自的定义,并研究互相之间的关系。

如果不特别说明,我们下面均设 $\set{X_n}_{n\ge 1}$ 和 $X$ 为概率空间 $(\Omega,\@F,\bb P)$ 上的随机变量。

\subsection{几乎必然收敛 (almost surely convergence)}

也叫做“以概率 $1$ 收敛”,即存在一个可测集 $\Omega'\subseteq\Omega$ 满足 $\Pr{\Omega'} = 1$,并且 $\forall \omega\in\Omega'$,$\lim_{n\to\infty}X_n(\omega) = X(\omega)$。我们可以等价的写作
\[
    \Pr{\lim_{n\to\infty} X_n=X} = 1.
\]

我们一般记作 $X_n\overset{a.s.}{\to} X$.

\subsection{依概率收敛(converge in probability)}

指的是对于任何 $\eps>0$,
\[
    \lim_{n\to\infty} \Pr{\abs{X_n-X}>\eps} = 0.
\]

我们一般记作 $X_n\overset{P}{\to} X$.

\subsection{依 $L^p$ 收敛(converge in $L^p$ )}

这里 $p\ge 1$ 是一个整数。它的定义是
\[
    \lim_{n\to\infty} \E{\abs{X_n-X}^p} = 0.
\]

我们一般记作 $X_n\overset{L^p}{\to} X$.

\subsection{依分布收敛(converge in distribution)}

依分布收敛和上述几种收敛模式不一样,它不要求这些随机变量生活在同一个概率空间中。我们可以假设 $X_n$ 的分布函数是 $F_n$,$X$ 的分布函数是 $F$。它的定义是,对于每一个 $F(x)$ \emph{连续}的点 $x$,有
\[
    \lim_{n\to\infty} F_n(x) = F(x).
\]

我们一般记作 $X_n\overset{D}{\to} X$.

\section{收敛之间的关系}

给出了这么多收敛的定义,小朋友一定有一堆疑问。为什么要有这么多不同种类的收敛?它们之间有什么关系?各自的直观又是什么?

对于第一个问题,我们在未来的学习中会逐渐体会到。但一个主要原因是,这些收敛有强有弱,我们关心的概率结论,在有的时候往往只能在比较弱的收敛意义下成立,或者在强的意义下成立需要更多的限制条件,或者更复杂的证明。我们现在来回答后两个问题。

我们假设 $q\ge p\ge 1$ 是整数。我们在下图中展示了各个收敛模式的关系。

\begin{figure}[h]
\centering
\includegraphics[width=0.6\textwidth]{figure/Figure18.1.png}
\end{figure}

\subsection{$\overset{a.s}{\to}$ 与 $\overset{P}{\to}$}

首先我们通过一个例子说明,$\overset{P}{\to}$ 不能推出 $\overset{a.s.}{\to}$。这个例子如图所示:设 $X_n$ 和 $X$ 都是定义在 $[0,1]$ 的均匀测度上的随机变量,其中 $X\equiv 0$。对于任意的 $n\ge 1$,我们设 $m=\lceil\log_2(n+1)\rceil-1$。则我们知道 $n\in [2^m,2^{m+1}-1]$。我们取 $k = n-(2^m-1)\in [1,2^m]$。我们定义
\[
    X_n =
    \begin{cases}
        1, & \mbox{$\omega \in [(k-1)\cdot 2^{-m},k\cdot 2^{-m})$}\\
        0, & \mbox{otherwise.}
    \end{cases}
\]

\begin{figure}[h]
\centering
\includegraphics[width=0.4\textwidth]{figure/Figure18.2.png}
\end{figure}

直观上说,我们让 $X_n$ 在图中为红色的线段部份取值为 $1$,其余的部分为 $0$。令 $X=0$。可以看出,红色部分是越来越少的,即 $X_n\ne X$ 的测度越来越小。即
\[
    \lim_{n\to\infty} \Pr{\abs{X_n-X}>\eps} = 0.
\]

但是,显然我们有 $X_n\overset{a.s}{\to} X$ 不成立。事实上,对于任意的 $\omega\in [0,1]$,我们都有 $X_n(\omega)$ 是不收敛的,因为红色的部分会无穷次的扫过 $\omega$ 这个点。

这个反例很好的展示了这两种收敛模式的区别:即 $X_n$ 与 $X$ 不同的地方测度尽管越来越小,但是这个位置是可以移动,这种移动阻止了几乎处处收敛。

我们接着来说明,$\overset{a.s.}{\to}$ 可以推出 $\overset{P}{\to}$。我们来证明,
\[
    X_n\overset{a.s}{\to} X \iff \forall \eps>0,\;\lim_{n\to\infty} \Pr{\sup_{k\ge n} \abs{X_k-X}>\eps} = 0.
\]

由于 $\abs{X_n-X}>\eps\implies \sup_{k\ge n} \abs{X_k-X}>\eps$,所以说明了 $X_n\overset{P}{\to} X$。对于每一个 $n$, 我们定义 $Z_n = \sup_{k\ge n} \abs{X_k-X}$。注意到
\[
\begin{aligned}
X_n\overset{a.s}{\to} X 
&\iff \exists \Omega'\subseteq\Omega \mbox{ s.t. } \Pr{\Omega'}=1 \mbox{ and }\forall \omega\in \Omega',\;X_n(\omega)\to X(\omega)\\
&\iff \exists \Omega'\subseteq\Omega \mbox{ s.t. } \Pr{\Omega'}=1 \mbox{ and }\forall \omega\in\Omega',\;Z_n(\omega)\to 0\\
&\iff \forall \eps\in\bb Q_{>0}, \exists \Omega'\subseteq\Omega \mbox{ s.t. } \Pr{\Omega'}=1 \mbox{ and }\forall \omega\in\Omega',\;\lim_{n\to\infty} Z_n(\omega)\le \eps\\
&\iff \forall \eps\in\bb Q_{>0}, \Pr{\lim_{n\to\infty}\set{\omega\cmid Z_n(\omega)>\eps}}=0\\
&\iff \forall \eps>0, \lim_{n\to\infty} \Pr{Z_n>\eps}=0.
\end{aligned}
\]

\subsection{$\overset{L^p}{\to}$ 与 $\overset{P}{\to}$}

我们接着说明,对于 $p\ge 1$,“依 $L^p$ 收敛”可以推出“依概率收敛”,但是反过来不成立。我们只需要对 $p=1$ 的情况进行证明,因为,我们接着马上要说明,如果 $p>1$,那么“依 $L^p$ 收敛”可以推出“依 $L^1$ 收敛”。

这件事情正确的直观也很容易,“依概率收敛”是说的随机变量 $X_n$ 和 $X$ 不一样的位置的测度趋向于 $0$。而“依 $L^p$ 收敛”要求的是在不一样的地方,这个测度还要乘上“两者所差”的值之后依旧趋向于 $0$。因此,这个要求更强一些。证明使用马尔可夫不等式即可:对于任何 $\eps>0$,
\[
    \lim_{n\to\infty} \Pr{\abs{X_n-X}>\eps}\le \lim_{n\to\infty} \frac{\E{\abs{X_n-X}}}{{\eps}} = 0.
\]

反过来不成立的例子是我们很熟悉的:设 $X_n$, $X$ 均是定义在 $(0,1)$ 的均匀测度上的随机变量。我们令 $X_n = n\cdot\bb I_{(0,\frac{1}{n})}$,$X=0$.

\subsection{$\overset{L^q}{\to}$ 与 $\overset{L^p}{\to}$}

和上面相同的直观指出,如果 $q>p$,那么 $X_n\overset{L^q}{\to} X$ 可以推出 $X_n\overset{L^p}{\to} X$。证明如下:
\[
\lim_{n\to\infty} \E{\abs{X_n-X}^p} = \lim_{n\to\infty} \E{\tp{\abs{X_n-X}^q}^{\frac{p}{q}}}\le \lim_{n\to\infty}\tp{\E{\abs{X_n-X}^q}}^{\frac{p}{q}} = 0.
\]

其中上式的不等号是利用琴生不等式以及 $f(x) = x^{\frac{p}{q}}$ 是一个 concave 函数的事实。

反过来不成立的例子可以类似前一种情况给出,留作练习。

\subsection{$\overset{P}{\to}$ 与 $\overset{D}{\to}$}

根据定义就可以知道,“依分布收敛” $\overset{D}{\to}$ 是一个很弱的概念,它关心的是分布函数的收敛性,甚至都不要求随机变量们生活在同一个概率空间上。

固定概率空间 $(\Omega,2^\Omega,\bb P)$ 为在 $\Omega=\set{0,1}$ 上的均匀分布。对于任意 $n$, 定义 $X_n(0)=0, X_n(1)=1$。定义 $X(0)=1, X(1)=0$。那显然 $X_n\overset{D}{\to} X$ 但是 $X_n\not\overset{P}{\to} X$.

我们现在来说明 $X_n\overset{P}{\to}X$ 可以推出 $X_n\overset{D}{\to} X$。我们想把事件 $\abs{X_n-X}>\eps$ 与 $X_n$ 的分布函数联系起来。我们使用下面一个基本的事实:对于任意 $\eps>0$,任意两个随机变量 $X,Y$ 和实数 $a$:
\[
Y\le a \implies X\le a+\eps \mbox{ or } \abs{Y-X}>\eps,
\]

即如果已知 $Y$ 不大于 $a$,则要么 $X$ 不大于 $a+\eps$,要么 $X$ 和 $Y$ 的差距比较大。我们使用这个不等式两次,并使用 union-bound,可以得到
\[
\begin{aligned}
    \Pr{X_n\le a} &\le \Pr{X\le a+\eps} + \Pr{\abs{X_n-X}>\eps}\\
    \Pr{X<a-\eps} &\le \Pr{X_n\le a}+ \Pr{\abs{X_n-X}>\eps}
\end{aligned}
\]

这便得到了
\[
    \Pr{X\le a-\eps}-\Pr{\abs{X_n-X}>\eps}\le \Pr{X_n\le a}\le \Pr{X\le a+\eps}+\Pr{\abs{X_n-X}>\eps}.
\]

我们让 $n$ 趋向于无穷大并让 $\eps\to 0$ 便得到了想要的结论。

\subsection{$\overset{a.s.}{\to}$ 与 $\overset{L^1}{\to}$}

这两者一般来说是不可比较的。事实上,在一定条件下,我们有 $X_n\overset{a.s.}{\to} X \implies X_n\overset{L^1}{\to} X$。如果我们存在一个可积的随机变量 $Y$,满足对于每一个 $n$,$\abs{X_n}\le Y$ 并且 $\abs{X}\le Y$。那么 容易验证 $\abs{X_n-X}\le 2Y$。显然我们也有 $\abs{X_n-X}\overset{a.s.}{\to}0$。因此由 DCT
\[
    \lim_{n\to\infty} \E{\abs{X_n-X}} = \E{\lim_{n\to\infty}\abs{X_n-X}} = 0.
\]

\section{集合的极限}

我们之前定义过集合的极限的概念。如果 $\set{A_n}_{n\ge 1}$ 是单调递增的( $\forall n, A_n\subseteq A_{n+1}$ ),那么
\[
    \lim_{n\to\infty} A_n \defeq \bigcup_{n\ge 1}A_n.
\]

类似的,如果 $\set{A_n}_{n\ge 1}$ 是单调递减的( $\forall n, A_n\supseteq A_{n+1}$ ),那么
\[
    \lim_{n\to\infty} A_n\defeq \bigcap_{n\ge 1}A_n.
\]

这可以类比于数列的极限。假设我们有一列实数 $a_n$,如果它是单调的数列,那么它一定存在极限(允许极限是正负无穷大的话)。而如果数列不单调的话,那么极限就\emph{不一定}存在了。但是,我们可以定义它的上极限和下极限:
\[
\begin{aligned}
    \limsup_{n\to\infty} a_n &\defeq \lim_{n\to\infty}\tp{\sup_{k\ge n} a_k}\\
    \liminf_{n\to\infty} a_n &\defeq \lim_{n\to\infty}\tp{\inf_{k\ge n} a_k}.
\end{aligned}
\]

上极限和下极限总是存在的,这是因为 $\tp{\sup_{k\ge n} a_k}_{n\ge 1}$ 与 $\tp{\inf_{k\ge n} a_k}_{n\ge 1}$ 分别是单调递减和单调递增的数列。我们尝试类似的定义\emph{集合列}的上极限与下极限。设 $\tp{A_n}_{n\ge 1}$ 是一列(不一定单调的)集合。我们定义
\[
\begin{aligned}
    \limsup_{n\to\infty} A_n &\defeq \lim_{n\to\infty}\tp{\sup_{k\ge n} A_k}\\
    \liminf_{n\to\infty} A_n &\defeq \lim_{n\to\infty}\tp{\inf_{k\ge n} A_k}.
\end{aligned}
\]

当然,我们还没有说 $\sup_{k\ge n} A_k$ 和 $\inf_{k\ge n} A_k$ 是怎么定义的。但是,我们可以很自然的想到,对于一个集族 $\set{B_n}_{n\in I}$,其上确界应该是包含每一个 $B_n$ 的最小的集合,而下确界应该是被每一个 $B_n$ 包含的最大的集合。因此
\[
\begin{aligned}
    \sup_{n\in I} B_n &\defeq \bigcup_{n\in I} B_n\\
    \inf_{n\in I} B_n &\defeq \bigcap_{n\in I} B_n.
\end{aligned}
\]

使用这个定义,以及对于单调集合族极限的定义,我们有:
\[
\begin{aligned}
    \limsup_{n\to\infty} A_n &\defeq \lim_{n\to\infty}\tp{\sup_{k\ge n} A_k} = \bigcap_{n\ge 1}\bigcup_{k\ge n} A_k\\
    \liminf_{n\to\infty} A_n &\defeq \lim_{n\to\infty}\tp{\inf_{k\ge n} A_k} = \bigcup_{n\ge 1}\bigcap_{k\ge n} A_k.
\end{aligned}
\]

另外一个比较重要的事情是我们来看看 $\limsup A_n$ 和 $\liminf A_n$ 究竟包含的哪些元素。简单的思考之后(记得思考哦),我们可以发现
\[
\begin{aligned}
    \limsup A_n &= \set{x\cmid x\mbox{ 在无穷多个 }A_n\mbox{ 中出现过}},\\
    \liminf A_n &= \set{x\cmid x\mbox{ 只在有限个 }A_n\mbox{ 中没出现过}}.
\end{aligned}
\]

基于这种直观含义,我们有的时候会把 “ $\limsup A_n$ ” 记作 “ $A_n$ i.o”,其中“i.o”是 “infinitely often” 的意思。

我们使用定义以及集合的 De-Morgan 律,可以马上得到
\[
\limsup A_n = \tp{\liminf A_n^c}^c.
\]

\section{波莱尔-坎泰利引理 (\href{https://en.wikipedia.org/wiki/Borel\%E2\%80\%93Cantelli_proposition}{\underline{Borel-Cantelli proposition}})}

我们接着介绍一个很常用的工具。它通常处理的问题是这样的:假设在一个固定的概率空间里,我们有一些坏事件 $\set{A_n}_{n\ge 1}$。我们想知道,有多大的概率,这些坏事件不会总发生。

\begin{proposition}[Borel-Cantelli]
如果 $\sum_{n\ge 1} \Pr{A_n}<\infty$,那么 $\Pr{A_n\mbox{ i.o}} = 0$.
\end{proposition}

我们前面刚说过 
\[
A_n \mbox{ i.o } = \limsup_n A_n = \set{\omega\cmid \omega\mbox{ 在无穷多个 }A_n\mbox{ 中出现过}}.
\]

因此,Borel-Cantelli 说的是,如果所有的坏事件(它们可能互相相关)发生的概率之和是一个有限数的话,那么,几乎一定(almost surely)这些坏事件不会不停发生。

Borel-Cantelli 的证明非常简单:
\[
\Pr{A_n\mbox{ i.o}} = \Pr{\lim_n\tp{\sup_{k\ge n} A_k}} = \lim_{n}\Pr{\sup_{k\ge n} A_k}\le \lim_n\sum_{k\ge n}\Pr{A_k}.
\]

上面式子里第二个等号是因为概率测度的连续性,不等号是使用了 union-bound。根据我们的条件,$\sum_{n\ge 1}\Pr{A_n}<\infty$,而一个收敛级数的 tail 一定是 $0$。所以我们有 $\Pr{A_n\mbox{ i.o}} = 0$.

Borel-Cantelli 反过来就不一定正确了,也就是说如果 $\sum_{n\ge 1} \Pr{A_n} = \infty$,不一定有 $\Pr{A_n\mbox{ i.o}}>0$。但是,如果这些坏事件是相互独立的,那么 $\Pr{A_n\mbox{ i.o}}=1$。这个结论又被称为第二 Borel-Cantelli 引理。

\begin{proposition}[Second Borel-Cantelli]
如果 $A_n$ 相互独立,那么
\begin{enumerate}
\item $\mathbb{P}(A_n \text{ i.o.}) = 0 \iff \sum_{n \geq 1} \mathbb{P}(A_n) < \infty$.
    \item $\mathbb{P}(A_n \text{ i.o.}) = 1 \iff \sum_{n \geq 1} \mathbb{P}(A_n) = \infty$.
\end{enumerate}
\end{proposition}

这个引理也说明,概率 $\Pr{A_n\mbox{ i.o.}}$ 只有 $0$ 或者 $1$ 两种取值。这实际上是一种更一般的现象,被称为 $0$-$1$ 律,我们在未来会介绍。

我们现在证明 Second Borel-Cantelli。事实上,我们只要证明 $\sum_{n\ge 1} \Pr{A_n}=\infty\implies \Pr{A_n\mbox{ i.o.}}=1$ 就可以了(why)。于是,我们利用独立的条件和 De-Morgen 律可以得到
\[
\Pr{\limsup_n A_n}=1-\Pr{\liminf_n A_n^c}=1-\lim_n\Pr{\bigcap_{k\ge n} A_k^c}=1-\lim_n \prod_{k\ge n}\Pr{A_k^c}.
\]

如果我们设 $x_k\defeq \Pr{A_k}$,那么
\[
    \Pr{\limsup_n A_n}=1-\lim_n \prod_{k\ge n}(1-x_k)\ge 1-\lim_n e^{-\sum_{k\ge n} x_k}=1.
\]

其中最后一个等号是因为 $\sum_{n} x_n=\infty$ 是一个发散的级数(因此它的 tail 是发散的)。

我们现在来使用 Borel-Cantelli 来证明一个有用的结论。即如果 $X_n\overset{P}{\to} X$,那么存在一个子序列 $n_1,n_2,\dots$,满足 $X_{n_k}\overset{a.s.}{\to} X$.

大家可以先想想,在我们前面说明 $X_n\overset{P}{\to} X\not\overset{a.s.}{\to} X$ 的例子里,这样一个子序列如何挑。

\begin{small}
~~我们只用挑那些红色都在最左边的 $X_{n_k}$ 即可。~~
\end{small}

以下的证明,是 Borel-Cantelli 引理的一个典型应用。对于每一个 $k\ge 1$,我们选取 $n_k$ 满足 $\Pr{\abs{X_{n_k}-X}\ge \frac{1}{k}}\le \frac{1}{2^k}$。由于 $X_n\overset{P}{\to} X$,这样的 $n_k$ 总是可以挑出来。我们用 $A_k$ 来表示坏事件“ $\abs{X_{n_k}-X}\ge \frac{1}{k}$ ”。那么根据定义 $\sum_{k\ge 1}\Pr{A_k} \le \sum_{k\ge 1} 2^{-k}<\infty$。于是使用 Borel-Cantelli,我们可以得到 $\Pr{A_n\mbox{ i.o.}} = 0$.

我们需要仔细解读一下 $\Pr{A_n\mbox{ i.o.}}=0$ 意味着什么。它说明,存在 $\Omega'\subseteq\Omega$,满足 $\Pr{\Omega'}=1$,对于任何 $\omega\in\Omega'$,只存在有限个 $k$,使得 $\abs{X_{n_k}(\omega)-X(\omega)}\ge \frac{1}{k}$ 成立。这意味着对于每一个这样的 $\omega\in \Omega'$,$X_{n_k}(\omega)\to X(\omega)$.

我们可以使用这个结论加强我们的老熟人控制收敛定理:

\begin{theorem}[控制收敛定理]
设 $X_n$ 为一列随机变量,满足 $\lim_{n \to \infty} X_n = X$ a.e.。如果存在一个随机变量 $Y$,满足
\begin{enumerate}
\item 对所有 $n \in \mathbb{N}$, $|X_n| \leq Y$;
    \item $Y$ 是可积的。
\end{enumerate}
那么 $\lim_{n \to \infty} \mathbf{E}[X_n] = \mathbf{E}[X]$。
\end{theorem}

我们现在说明,我们可以把条件里的 $X_n\overset{a.s.}{\to} X$ 弱化成 $X_n\overset{P}{\to}X$。我们使用反证法。假设 $\lim_n \E{X_n} = \E{X}$ 不成立。那么,一定存在一个子序列 $\set{n_k}_{k\ge 1}$ 满足 $\lim_{k\to\infty} \E{X_{n_k}} = L\ne \E{X}$。根据条件,我们知道 $X_{n_k}\overset{P}{\to} X$。因此,使用我们刚才证明的结论,从 $\set{n_k}_{k\ge 1}$ 中我们能再找到一个子序列 $\set{m_j}_{j\ge 1}\subseteq \set{n_k}_{k\ge 1}$,满足 $X_{m_j}\overset{a.s.}{\to} X$。根据 a.s. 版本的 DCT,我们知道 $\lim_{j\to\infty} \E{X_{m_j}} = \E{X}\ne L$,这与 $\set{X_{m_j}}$ 是 $\set{X_{n_k}}$ 的子序列矛盾,因为 $\E{X_{m_j}}$ 与 $\E{X_{n_k}}$ 理应有一样的极限。

\newpage