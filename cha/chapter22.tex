\chapter{依分布收敛,De Moivre 中心极限定理}

\section{中心极限定理的动机}

我们之前讨论了大数定律,其想说的事情是给定一系列随机变量 $X_1,X_2,\dots$,如果每一个 $X_i$ 的期望均是 $\mu$,那么在某些条件下,其部分和 $S_n = \sum_{i\in [n]} X_i$ 满足 $\frac{S_n}{n}$ 会收敛到 $\mu$。注意到,每一个 $\frac{S_n}{n}$ 都是一个随机变量,而 $\mu$ 本身是一个固定数(常值随机变量)。为什么随机变量会收敛到一个固定的数呢?如果我们再假设每一个 $X_i$ 满足 $\Var{X_i}\le \sigma^2$,并且它们是相互独立的,那么我们知道
\[
    \Var{\frac{S_n}{n}} = \frac{\sum_{i=1}^n\Var{X_i}}{n^2}\le \frac{\sigma^2}{n}\to 0.
\]
也就是说,我们把 $S_n$ 除 $n$ 的操作使得当 $n$ 足够大时,$\frac{S_n}{n}$ 的方差趋向于零了,因此,其也自然的收敛到一个常数。

一个很自然的问题是,对于任意一个关于 $n$ 的递增函数 $f(n)$,随机变量 $\frac{S_n}{f(n)}$ 的收敛情况是怎么样的呢。简单计算就知道,当 $f(n) = \omega(\sqrt{n})$ 的时候,$\Var{\frac{S_n}{f(n)}}$ 同样收敛到 $0$;而当 $f(n) = o(\sqrt{n})$ 的时候,其方差会趋向于无穷大。对于这样的 $f(n)$,有一些有趣的性质可以讨论(我们在作业中会遇到),但最有意思的事情发生在 $f(n)=\sqrt{n}$ 的时候,这便是中心极限定理所讨论的问题。

我们先严格化我们的设定。假设 $X_1,X_2,\dots$ 是定义在同一个概率空间上的独立同分布的随机变量,满足 $\E{X_1}=\mu$,$\Var{X_1}=\sigma^2$ 均是有限的。我们关心 $\frac{S_n}{\sqrt{n}}$ 的极限行为。我们首先证明一个有一些惊人的结论:

\begin{proposition}
如果 $\E{X_1}=0$ 并且 $\E{X_1^4}<\infty$,那么不存在一个随机变量 $X$ 使得 $\frac{S_n}{\sqrt{n}}\overset{P}{\to} X$。
\end{proposition}

我们假设这样一个 $X$ 存在。前面的课上证明过 $\frac{S_n}{\sqrt{n}}\overset{P}{\to} X$ 可以推出一定存在一个子序列 $\set{n_j}$ 满足 $\frac{S_{n_j}}{\sqrt{n_j}}\overset{a.s.}{\to} X$。由于我们知道对于每一个 $j$,$\E{\frac{S_{n_j}}{\sqrt{n_j}}} = 0$ 并且 $\Var{\frac{S_{n_j}}{\sqrt{n_j}}}=\sigma^2$。下面的引理可以保证 $X$ 一定也满足 $\E{X} = 0$,$\Var{X}=\sigma^2$。这是一个类似 DCT 和 MCT 的 yet another 保证极限和期望可以交换的充分条件,我们把它的证明放在本次讲义最后(事实上,极限和期望交换的充要条件是所谓的“一致可积(\href{https://en.wikipedia.org/wiki/Uniform_integrability}{\underline{uniform integrability}})”,因为课时原因我们不再介绍)。

\begin{lemma}
设 $X_1,X_2,\dots$ 是一族随机变量满足 $X_n\overset{a.s}{\to} X$。如果存在 $\eps>0$ 和常数 $M$,使得对于每一个 $n$, $\E{\abs{X_n}^{1+\eps}}\le M$,那么
\[
\lim_{n\to\infty} \E{X_n} = \E{X}.
\]

\end{lemma}

但是另一方面,我们知道 $X$ 是一个所谓的“尾变量”,也就是说,对于任何 Borel 集 $B\in \@B(\bb R)$,事件 $[X\in B]$ 都是 $\@F^*_\infty$ 中的一个尾事件,这儿显然 $\@F_n = \sigma(X_1,\dots,X_n)$。根据 Kolmogorov 0-1 律,$[X\in B]$ 发生的概率要么是 $0$ 要么是 $1$。这也说明 $X$ 一定等于某一个常数。但这与它的方差是 $\sigma^2>0$ 矛盾。

上面的讨论说明,我们不能期待对于 $\frac{S_n}{\sqrt{n}}$ 在依概率收敛或者几乎处处收敛的意义上说什么。我们考虑一个更弱的收敛定义,即依分布收敛。回忆我们之前定义过的依分布收敛:设 $X_1,X_2,\dots$ 分别有分布函数 $F_n$,并且 $X$ 有分布函数 $F$。我们说 $X_n\overset{D}{\to} X$ 当且仅当对于 $F$ 的每一个连续的点 $x$,有 $\lim_{n\to\infty} F_n(x) = F(x)$。最基本的中心极限定理便是如下结论:

\begin{theorem}
如果独立同分布的随机变量 $X_1,X_2,\dots$ 满足 $\E{X_1}=\mu, \Var{X_1}=\sigma^2$ 均为有限的,那么
\[
\frac{S_n-n\mu}{\sigma\sqrt{n}}\overset{D}{\to} Y\sim\+N(0,1).
\]
\end{theorem}

换句话说,如果我们把 $X_i$ 归一化成期望为 $0$ 方差为 $1$ 的随机变量(即 $X_i' = \frac{X_i-\mu}{\sigma}$ ),并令 $S_n'=\sum_{i=1}^n X_i'$ 为归一化后的部分和。那么 $\frac{S_n'}{\sqrt{n}}$ 依分布收敛到标准正态分布。
\[
    \frac{\sum_{i=1}^n X_i'}{\sqrt{n}}\overset{D}{\to} Y\sim\+N(0,1).
\]

相比大数定律,我认为这是一个非常让人意外的结果,因为它并没有对 $X_i$ 的分布有要求,只规定了它的期望和方差。而中心极限定理告诉我们,不管 $X_i$ 本身的分布是什么,当足够多的独立的 $X_i$ 加在一起的时候,一定呈现出正态分布的样子。这也是正态分布在我们生活中经常出现的原因。~~但我上课试图解释为什么教务处总希望同学的成绩是正态分布但发现好像不管怎么model正态分布需要的条件都不太成立所以教务处为什么~~

\section{棣莫弗-拉普拉斯中心极限定理(\href{https://en.wikipedia.org/wiki/De_Moivre\%E2\%80\%93Laplace_theorem}{\underline{De Moivre–Laplace theorem}})}

我们今天先证明一个简单版本的中心极限定理,即当每一个 $X_i$ 都具有独立同分布的伯努利分布的情形。它被称作棣莫弗-拉普拉斯中心极限定理。我们证明的方法也很暴力,就是直接计算并估计 $\frac{S_n}{\sqrt{n}}$ 的分布函数。在这个证明中使用的一些估计技巧是非常有用并且常见的。

我们严格的陈述一下要证明的结论。设 $X_1,X_2,\dots$ 是独立的满足 $\!{Ber}(p)$ 分布的随机变量,其中 $p\in (0,1)$ 是一个常数,那么 
\[
    \frac{S_n-pn}{\sqrt{(p-p^2)n}} \overset{D}{\to} Y\sim \+N(0,1).
\]

为了方便,我们只证明 $p=\frac{1}{2}$ 的情况,对于一般的 $p$ 可以完全类似的证明。

回忆我们用 $\phi(x) = \frac{1}{\sqrt{2\pi}}e^{-\frac{x^2}{2}}$ 来表示标准正态分布 $\+N(0,1)$ 的概率密度函数。事实上,我们知道 $S_n$ 是满足 $\!{Binom}(n,\frac{1}{2})$ 分布的。因此,我们可以直接计算出它的概率质量函数 $p_n(k) \defeq \Pr{S_n=k} = \binom{n}{k}2^{-n}$。我们实际上想证明的是,当 $n$ 足够大的时候,$p_n(k)$ 所决定的离散的点列(图1),和 $\phi(x)$ 对应的曲线(图2),在做适当放缩之后是逐渐趋向于一致的(图3)。

\begin{figure}[h]
\centering
\includegraphics[width=0.7\textwidth]{figure/Figure22.1.png}
\end{figure}

便让我们来严格证明这件事情。我们首先说明,当 $k$ 比较接近其平均值 $\frac{n}{2}$ 时, $p_n(k)$ 的值和进行合适放缩后的 $\phi$ 的值的差距是非常小的。

\begin{lemma}
设 $C>0$ 是一个常数。那么
\[
\max_{\abs{k-n/2}\le C\cdot \sqrt{n}} \abs{\frac{p_n(k)\cdot \sqrt{n/4}}{\phi\tp{\frac{k-n/2}{\sqrt{n/4}}}}-1} = \+O\tp{n^{-1}}.
\]
\end{lemma}

我们先解释一下上面引理里是如何缩放 $\phi(x)$ 的。对于 $k\sim\!{Binom}(n,\frac{1}{2})$,我们知道其期望是 $n/2$,方差是 $n/4$。因此 $\frac{k-n/2}{\sqrt{n/4}}$ 就是一个期望为 $0$ 方差为 $1$ 的随机变量,所以 $p_n(k)$ 应该是和 $\phi\tp{\frac{k-\frac{n}{2}}{\sqrt{n/4}}}$ 相比较。但此时,我们需要把它除掉 $\sqrt{n/4}$ 才能保证其(对 $k$ 的)积分为 $1$。

我们证明引理的主要工具就是\href{https://en.wikipedia.org/wiki/Stirling\%27s_approximation}{\underline{斯特林公式}}:$n! = \sqrt{2\pi n}\tp{\frac{n}{e}}^n\tp{1+\+O\tp{n^{-1}}}$。使用这个公式以及 $k\approx \frac{n}{2}$ 的事实,我们可以得到
\[
\begin{aligned}
    p_n(k)
    &=\binom{n}{k}2^{-n}\\
    &=\frac{n!}{k!(n-k)!}2^{-n}\\
    &=\frac{\sqrt{2\pi n}}{\sqrt{2\pi k}\sqrt{2\pi (n-k)}} \frac{n^n}{k^k(n-k)^{n-k}}\cdot 2^{-n}\cdot\tp{1+\+O\tp{n^{-1}}}\\
    &=\frac{1}{\sqrt{2\pi n}}\cdot \frac{1}{\sqrt{k/n}\cdot \sqrt{1-k/n}}\cdot \frac{2^{-n}}{\tp{k/n}^k\tp{1-k/n}^{n-k}}\cdot\tp{1+\+O\tp{n^{-1}}}.
\end{aligned}
\]

我们的目标是把上式和 $\phi\tp{\frac{k-n/2}{\sqrt{n/4}}} = \exp\tp{-\frac{1}{2n}\cdot (2k-n)^2}$ 进行比较。我们的策略如下:我们知道 $k\approx \frac{n}{2}$,所以我们把分母凑出形如 $2k/n\approx 1$ 的项,然后尝试把这些接近于 $1$ 的项用 $1+x\approx e^x$ 换成指数形式,并期待能够变成 $\phi(\cdot)$ 的形式。因此,我们可以继续变形得到
\[
\begin{aligned}
    p_n(k)
    &=\sqrt{\frac{2}{\pi n}} \cdot \frac{1}{\sqrt{1+\frac{2k-n}{n}}\cdot\sqrt{1-\frac{2k-n}{n}}} \cdot \frac{1}{\tp{1+\frac{2k-n}{n}}^k\cdot \tp{1-\frac{2k-n}{n}}^{n-k}}\cdot\tp{1+\+O\tp{n^{-1}}}.
\end{aligned}
\]

注意到,我们目标 $\phi(\cdot)$ 里出现的是形如 $e^{-(2k-n)^2}$ 的项,其指数上关于 $(2k-n)$ 的依赖是二次的。我们仅仅使用 $1\pm x \approx e^{\pm x}$ 这个一阶近似是不够的,所以我们来计算 $1\pm x$ 的二阶近似。

\begin{small}
事实上我们注意到由于 $k\approx n/2$,所以 $\tp{1+\frac{2k-n}{n}}^k\cdot \tp{1-\frac{2k-n}{n}}^{n-k} \approx \tp{1-\tp{\frac{2k-n}{n}}^2}^k\approx e^{-k\tp{\frac{2k-n}{n}}^2}$。直接估计这儿 $\approx$ 的误差也可以,但我认为下面的技巧更有通用性。
\end{small}

使用 $\log (1+x)=x-\frac{x^2}{2}+\+O\tp{x^3}$,我们有 $1+x = e^{\log\tp{1+x}} = e^{x-\frac{x^2}{2}}\cdot \tp{1+\+O\tp{\abs{x}^3}}$。注意我们这里要利用 $\abs{k-\frac{n}{2}}=C\cdot \sqrt{n}$ 仔细估计误差项。可以得到
\[
    p_n^k = \sqrt{\frac{2}{\pi n}}\cdot e^{\frac{1}{2}\cdot \tp{\frac{2k-n}{n}}^2}\cdot e^{-\frac{1}{2}\cdot \frac{(2k-n)^2}{n}}\tp{1+\+O\tp{n^{-1}}} = \sqrt{\frac{2}{\pi n}}\cdot e^{-\frac{1}{2}\cdot \frac{(2k-n)^2}{n}}\cdot \tp{1+\+O\tp{n^{-1}}}.
\]

而这正是 $\frac{\phi\tp{\frac{k-n/2}{\sqrt{n/4}}}}{\sqrt{n/4}}\tp{1+\+O\tp{n^{-1}}}$。

设 $F_n$ 是 $\frac{S_n-\frac{n}{2}}{\sqrt{n/4}}$ 的分布函数。我们接下来只需要说明(why?)对于任何固定的常数 $a<b\in \bb R$,我们有 $\lim_{n\to\infty} F_n(b)-F_n(a) = \int_{a}^b \phi(x)\dd x$。

注意到
\[
\begin{aligned}
    \abs{F_n(b)-F_n(a)-\int_a^b\phi(x)\dd x}
    &\le \abs{F_n(b)-F_n(a) - \sum_{k\in \bb N\colon a\le \frac{k-n/2}{\sqrt{n/4}}\le b}\frac{\phi\tp{\frac{k-n/2}{\sqrt{n/4}}}}{\sqrt{n/4}}}\\
    &\quad +\abs{\sum_{k\in \bb N\colon a\le \frac{k-n/2}{\sqrt{n/4}}\le b}\frac{\phi\tp{\frac{k-n/2}{\sqrt{n/4}}}}{\sqrt{n/4}} - \int_a^b\phi(x)\dd x}
\end{aligned}
\]

上式前一项的求和里最多有 $\+O(\sqrt{n})$ 项,根据我们前面证明的引理,这一项带来的误差是 $\frac{1}{\sqrt{n}}$ 级别。而上式第二项实际上是给出了积分和它的黎曼和的差,因此在 $n$ 趋向于无穷大的时候趋向于 $0$。这便证明了我们的结果。

\subsection{省略的证明}

我们现在来证明

\begin{lemma}
设 $X_1,X_2,\dots$ 是一族随机变量满足 $X_n\overset{a.s}{\to} X$。如果存在 $\eps>0$ 和常数 $M$,使得对于每一个 $n$, $\E{\abs{X_n}^{1+\eps}}\le M$,那么
\[
\lim_{n\to\infty} \E{X_n} = \E{X}.
\]
\end{lemma}

这个证明来自这个 \href{https://terrytao.wordpress.com/2015/10/03/275a-notes-1-integration-and-expectation/}{\underline{notes}}。

对于任意 $m>0$ 和随机变量 $Y$,我们定义一个新的随机变量 
\[
    Y^{[m]}(\omega) \defeq
    \begin{cases}
        m & \mbox{ if }Y(\omega)>m\\
        -m & \mbox{ if }Y(\omega)<-m\\
        Y(\omega) & \mbox{ if }Y(\omega)\in [m,-m].
    \end{cases}
\]

即 $Y^{[m]}$ 把 $Y$ 大于 $m$ 和小于 $-m$ 的部分分别换成 $m$ 和 $-m$。于是 $\abs{Y^{[m]}}\le m$。根据 DCT,我们显然有
\[
    \lim_{n\to\infty} \E{X_n^{[m]}} = \E{X^{[m]}}.
\]

根据定义,我们又有
\[
    \abs{X_n-X_n^{[m]}} \le \tp{\frac{\abs{X_n}}{m}}^\eps \abs{X_n} = m^{-\eps}\abs{X_n}^{1+\eps}.
\]

于是,$\E{X_n}=\E{X_n^{[m]}}\pm \+O\tp{m^{-\eps}M}$。

根据 Fatou 引理,我们有
\[
\E{\abs{X}^{1+\eps}} \le \E{\liminf\abs{X_n}^{1+\eps}} \le \liminf\E{X_n^{1+\eps}}=M.
\]

所以我们使用类似的推理可以得到
\[
    \E{X} = \E{X^{[m]}}\pm \+O\tp{m^{-\eps}M}.
\]

于是
\[
\limsup_{n\to\infty} \E{X_n}, \liminf_{n\to\infty}\E{X_n} = \lim_{n\to\infty} \E{X_n^{[m]}}\pm \+O\tp{m^{-\eps}M} = \E{X}\pm \+O\tp{m^{-\eps}M}.
\]

令 $m\to\infty$ 便得证。