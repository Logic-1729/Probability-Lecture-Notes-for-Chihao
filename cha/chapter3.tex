\chapter{事件的条件概率}

上节课我们介绍了概率空间 $(\Omega, \mathscr{F}, \mathbb{P})$ 的定义,以及它们需要满足的公理。很多关于概率空间的一些性质和操作,都可以解释成集合上对应的性质和操作。我们今天继续介绍一些性质,它们同样可以看成集合的性质,但本身也有很丰富的概率含义。可以看到,在实际应用中,我们实际思考的对象是这些集合后面的概率。

\section{条件概率及性质}

我们前面介绍过,概率空间,以及相关的记号,都是为了抽象出我们在日常做概率试验的时候的一些直观的事情。对于扔两个骰子的问题,我们可以问,在已知两个骰子的和是偶数的情况下,第一个骰子是 $6$ 的概率。经验告诉我们,这个概率应该是用“第一个是 $6$ 并且和为偶数”的概率,除上“和为偶数的概率”。数学上,我们把这个称之为 \textit{条件概率} ,是定义在两个事件 $A$ 和 $B$ 上,用来表示在已知 $B$ 发生的情况下 $A$ 发生的概率。我们用记号 $\mathbb{P}(A \mid B)$ 来表示,它仅在 $\mathbb{P}(B) > 0$ 的时候有定义。

\begin{definition}[事件的条件期望]
\[
\mathbb{P}(A \mid B) = \frac{\mathbb{P}(A \cap B)}{\mathbb{P}(B)}.
\]
\end{definition}

注意到这个是条件概率的定义式,我们可以把它写成
\[
\mathbb{P}(A \cap B) = \mathbb{P}(B) \cdot \mathbb{P}(A \mid B).
\]

这个形式可以更方便的把它想象成“事件 $A$ 和 $B$ 同时发生的概率,等于 $B$ 发生的概率,乘上在已知 $B$ 发生的时候 $A$ 发生的概率”。

对于 $n$ 个事件 $A_1, \dots, A_n$,我们不停地使用上式,可以得到
\[
\mathbb{P}\left( \bigcap_{i=1}^n A_i \right) = \prod_{k=1}^n \mathbb{P}\left( A_k \,\middle|\, \bigcap_{i=1}^{k-1} A_i \right).
\]

这个式子被称之为\href{https://en.wikipedia.org/wiki/Chain_rule_(probability)}{\underline{条件概率的链式法则}}。他可以读作:

\begin{leftbarquote}

    \quad \quad 如果我们想计算 $n$ 个事件同时发生的概率,我们先用第一个事件发生的概率,乘上在第一个事件发生的情况下第二个事件发生的概率,再乘上在第一第二个事件都发生的情况下,第三个事件发生的概率,\dots

\end{leftbarquote}

我们可以通过一些简单的例子来验证我们对于这个概念的理解。

\begin{example}
假设我有正好两个孩子,其中至少一个是女孩,那么两个都是女孩的概率是多大?
\end{example}

在计算一个概率问题的时候,总是要先对概率空间进行合适的建模。这个问题的概率空间我们可以取样本空间为四元组 $\Omega = \{FF, FM, MF, MM\}$,$\mathscr{F} = 2^\Omega$,$\mathbb{P}$ 为均匀分布。其中事件 $B := \text{“至少一个是女孩”} = \{FF, FM, MF\}$;事件 $A := \text{“两个都是女孩”} = \{FF\}$。

所以
\[
\mathbb{P}(A \mid B) = \frac{\mathbb{P}(A \cap B)}{\mathbb{P}(B)} = \frac{\mathbb{P}(\{FF\})}{\mathbb{P}(\{MF, FM, FF\})} = \frac{1/4}{3/4} = \frac{1}{3}.
\]


\begin{example}
假设面前有三个抽屉,每个抽屉里有两块硬币,分别是“金币+金币”,“金币+银币”,“银币+银币”。现在我随机选一个抽屉,并且随机在这个抽屉里选一个硬币,请问在选出的硬币是金币的情况下,和选出的硬币同一个抽屉的另一个也是金币的概率是多大?
\end{example}

在这个问题里,我们可以用 $\Omega = \{\text{一,二,三}\} \times\{1,2\}$,$\mathscr{F} = 2^\Omega$,$\mathbb{P}$ 为均匀分布来进行建模。对于样本点 $(i, j)$,我们想表达的意思是选出来的硬币是第 $i$ 个抽屉里的第 $j$ 个硬币(硬币的顺序就按照题干里说的那般)。那么定义事件
\[
B = \text{“选出的硬币是金币”} = \{(\text{一},1), (\text{一},2), (\text{二},1)\};
\]
\[
A = \text{“和选出的硬币同一个抽屉的另一个也是金币”} = \{(\text{一},1), (\text{一},2), (\text{二},2)\}.
\]

因此,我们有
\[
\mathbb{P}(A \mid B) = \frac{\mathbb{P}(A \cap B)}{\mathbb{P}(B)} = \frac{1/3}{1/2} = \frac{2}{3}.
\]


\section{独立性}

如果对于事件 $A$ 和 $B$,我们有 $\mathbb{P}(A \cap B) = \mathbb{P}(A) \cdot \mathbb{P}(B)$,或者等价的 $\mathbb{P}(A \mid B) = \mathbb{P}(A)$,我们就称 $A$ 和 $B$ 是独立的,有时候会记作 $A \perp B$。直观上,事件 $A$ 和 $B$ 独立表示 $A$ 或者 $B$ 是否发生,对于对方是否发生没有影响。比如我们扔两次骰子,为事件“第一次扔的是 $3$”,为事件“第二次扔的是 $4$”,则这两件事情是独立的。独立性,我认为,是一个非常具有概率风味的概念。尽管到目前为止,如果 $\Omega$ 是有限集,我们之前说过,所有的这些概率都可以用组合计数来研究,但独立性提供的直观,用计数的语言来说,并不是特别方便。

我们可以把独立的概念推广到更多的事件上去。我们说事件 $A_1, A_2, \dots, A_n$ 是\textbf{相互独立(mutually independent)}的,如果其满足对于任意的 $I \subseteq [n]$,

\[
\mathbb{P}\left( \bigcap_{i \in I} A_i \right) = \prod_{i \in I} \mathbb{P}(A_i).
\]

这个定义看起来非常强,因为它要求上述等式对于每一个 $I \subseteq [n]$ 都成立。

\begin{anymark}[注解]

\begin{enumerate}
    \item 如果只要求 $I = [n]$ 是不够的。假设 $n = 3$,$A_1 = A_2$,$A_3 = \varnothing$,则 $\mathbb{P}(A_1 \cap A_2 \cap A_3) = \mathbb{P}(A_1)\mathbb{P}(A_2)\mathbb{P}(A_3) = 0$。但显然这仨不一定独立。事实上,我们可以把独立的定义写成
\[
\mathbb{P}\left( \bigcap_{i=1}^n B_i \right) = \prod_{i=1}^n \mathbb{P}(B_i),
\]

其中 $B_i = A_i$ 或者 $\Omega$。

    \item 如果只要求等式对于 $I \in \binom{[n]}{2}$ 成立是不够的。一个经典的例子是概率空间为 $\Omega = \{HH, HT, TH, TT\}$ 上的均匀分布,即投两次均匀硬币的结果(H=Head,T=Tail)。我们考虑 $A_1 = \text{“第一次硬币出H”}$,$A_2 = \text{“第二次硬币出H”}$,$A_3 = \text{“两次硬币结果不一样”}$。可以验证,这三个事件对于任何 $I \in \binom{[3]}{2}$,是满足独立性定义的等式的,但对于 $I = [3]$ 并不满足。实际上,如果该等式仅对于 $I \in \binom{[n]}{2}$ 成立,我们称 $A_1, \dots, A_n$ 是\textbf{两两独立(pairwise independent)}的。这是一个在计算机科学里很重要的概念,原因在于,我们在设计算法的时候,独立的随机数本身是一个重要的资源,我们需要代价才能够产生它们。而很多问题里,对于独立的要求并没有像定义那么强,有的时候两两独立就足够满足我们的要求了。而生成两两独立的随机数,代价要小很多。
\end{enumerate}


\end{anymark}

对于无穷多个事件 $\{A_j\}_{j \in J}$,我们说它们相互独立,但且仅当对于 $J$ 的任何一个有限子集 $I$,$\{A_i\}_{i \in I}$ 相互独立。

\section{全概率公式}

假设事件 $B_1, B_2, \dots, B_n, \cdots \in \mathscr{F}$ 构成了样本空间的一个分划,即 $\Omega = \bigcup_{i \geq 1} B_i$ 并且对于 $i \ne j$,$B_i \cap B_j = \varnothing$。根据集合论的知识我们知道,对于任何集合 $A$,$A \cap B_1, A \cap B_2, \dots, A \cap B_n, \dots$ 也构成了 $A$ 的一个分划。如果我们取 $A \in \mathscr{F}$,则根据概率论的公理,我们有
\[
\mathbb{P}(A) = \mathbb{P}\left( \bigcup_{n \geq 1} (A \cap B_n) \right) = \sum_{n \geq 1} \mathbb{P}(A \cap B_n).
\]
这便是全概率公式(\href{https://en.wikipedia.org/wiki/Law_of_total_probability}{\underline{Law of total probability}})。我们可以把 $\mathbb{P}(A \cap B_n)$ 用条件概率写出来,即
\[
\mathbb{P}(A) = \sum_{n \geq 1} \mathbb{P}(B_n) \cdot \mathbb{P}(A \mid B_n).
\]

用人话说,就是想算 $A$ 的概率,可以先按照 $B_n$ 的情况进行分类再全部加起来,而每一类的概率是 $B_n$ 发生的概率乘上在 $B_n$ 发生的情况下 $A$ 发生的概率。

\section{一些例子}

我们接下来使用上述工具,来计算一些例子。

\begin{example}[抗原检测]
假设人群中感染新冠的概率是 $0.2\%$,而抗原检测的准确性是 $99\%$。现在你做了个检测发现阳了,那真的感染新冠的概率是多大?

\end{example}

这个问题的结论有时候会产生一些悖论。因为看起来抗原检测的准确性非常高,因此如果检测阳性,那应该大概率真的被感染了。但通过计算我们会发现不然。

首先我们用概率空间来进行建模。我们假设人群有 $N$ 个人,因此样本空间可以取 $\Omega = [N] \times \{\pm 1\} \times \{\pm 1\}$。这里面的每一个三元组 $(a, i, j)$,其中 $a$ 表示人的 id,$i$ 表示是否被感染,$j$ 表示是否抗原阳性。可以思考一下,我们如何设置 $\mathbb{P}$ 来满足题设的要求,这总是可以做到的。

我们来计算在已知抗原阳性的情况下,没有被感染的概率 $\mathbb{P}(\text{没感染} \mid \text{阳性})$。我们使用条件概率的定义,有
\[
\mathbb{P}(\text{没感染} \mid \text{阳性}) = \frac{\mathbb{P}(\text{没感染} \cap \text{阳性})}{\mathbb{P}(\text{阳性})}.
\]

我们希望把这个分式上下都写成我们题设里告诉能算的东西,因此需要用我们之前介绍的公式来改写一下。对于分子,我们有
\[
\mathbb{P}(\text{没感染} \cap \text{阳性}) = \mathbb{P}(\text{阳性} \mid \text{没感染}) \cdot \mathbb{P}(\text{没感染}).
\]

而对于分母,我们使用全概率公式,有
\[
\mathbb{P}(\text{阳性}) = \mathbb{P}(\text{阳性} \cap \text{感染}) + \mathbb{P}(\text{阳性} \cap \text{没感染}) = \mathbb{P}(\text{阳性} \mid \text{感染}) \cdot \mathbb{P}(\text{感染}) + \mathbb{P}(\text{阳性} \mid \text{没感染}) \cdot \mathbb{P}(\text{没感染}).
\]

这样一番操作之后,我们会发现每一个量我们都会计算了,即
\[
\mathbb{P}(\text{阳性} \mid \text{没感染}) = 1\%, \quad \mathbb{P}(\text{没感染}) = 99.8\%, \quad \mathbb{P}(\text{阳性} \mid \text{感染}) = 99\%, \quad \mathbb{P}(\text{感染}) = 0.2\%,
\]

把这些数字代进去,我们可以得到
\[
\mathbb{P}(\text{没感染} \mid \text{阳性}) \approx 83.4\%.
\]

也就是说,没事去测个抗原,即使阳性了,大概八成概率也没有被感染。

上面这个计算过程有时候也被称为贝叶斯公式(\href{https://en.wikipedia.org/wiki/Bayes%27_theorem}{\underline{Bayes' formula}})。

\begin{example}[双胞胎]
双胞胎有两种,同卵双生和异卵双生。一家医院想知道所有双胞胎中同卵双生的比例有多大,但是做这个检测的成本太高了。统计学家说,其实你只要统计一下每一对双胞胎的性别,就能推算出这个比例。我们知道,同卵双生子的性别一定是相同的,而异卵双生子的性别是独立的。因此,使用全概率公式,我们有
\[
\mathbb{P}(\text{同性}) = \mathbb{P}(\text{同性} \mid \text{同卵}) \cdot \mathbb{P}(\text{同卵}) + \mathbb{P}(\text{同性} \mid \text{异卵}) \cdot \mathbb{P}(\text{异卵}) = 1 \cdot \mathbb{P}(\text{同卵}) + 0.5 \cdot (1 - \mathbb{P}(\text{同卵})).
\]

因此,我们只要简单的统计出 $\mathbb{P}(\text{同性})$,即所有双胞胎中同性别的比例,便可以用上面的公式解出双胞胎中同卵双生的比例。
\end{example}

\begin{example}[生日悖论]
假设一个班级有 $n$ 位同学($n \ge 3$),每个人的生日都等可能地出现在一年中的 $m$ 天里,并且每个人的生日是相互独立的。我们定义事件 $A_{i,j}$ 表示第 $i$ 位同学和第 $j$ 位同学的生日在同一天。

这个看似简单的问题,其结论常常令人惊讶。直觉上,如果一年有 $m=365$ 天,那么需要很多人(比如超过100人)才能让“至少有两人同一天生日”的概率变得很大。但计算结果会告诉我们,其实只需要很少的人就能达到很高的概率。
\end{example}

我们先证明,任意两个不同的事件 $A_{i,j}$ 和 $A_{k,l}$ 是独立的。

对于任意 $i \ne j$,显然有 $\mathbb{P}(A_{i,j}) = \frac{1}{m}$,因为第 $j$ 个人的生日与第 $i$ 个人相同的概率是 $\frac{1}{m}$。

现在考虑两个事件 $A_{i_1,j_1}$ 和 $A_{i_2,j_2}$,其中 $(i_1, j_1) \ne (i_2, j_2)$。我们需要分两种情况讨论:

\textit{情况一:} 四个下标 $i_1, j_1, i_2, j_2$ 互不相同。
此时,事件 $A_{i_1,j_1}$ 和 $A_{i_2,j_2}$ 涉及的是四组完全独立的人。因此,
\[
\mathbb{P}(A_{i_1,j_1} \cap A_{i_2,j_2}) = \frac{m \cdot m \cdot m^{n-4}}{m^n} = \frac{1}{m^2} = \mathbb{P}(A_{i_1,j_1}) \cdot \mathbb{P}(A_{i_2,j_2}).
\]

\textit{情况二:} 两个事件共享一个下标,例如 $i_1 = i_2$ 且 $j_1 \ne j_2$。
此时,事件 $A_{i_1,j_1}$ 和 $A_{i_2,j_2}$ 意味着第 $i_1$ 个人的生日同时等于第 $j_1$ 个人和第 $j_2$ 个人的生日,即三个人生日相同。因此,
\[
\mathbb{P}(A_{i_1,j_1} \cap A_{i_2,j_2}) = \frac{m \cdot m^{n-3}}{m^n} = \frac{1}{m^2} = \mathbb{P}(A_{i_1,j_1}) \cdot \mathbb{P}(A_{i_2,j_2}).
\]

综上所述,任意两个不同的事件 $A_{i,j}$ 和 $A_{k,l}$ 都满足独立性的定义,因此事件族 $\{A_{i,j}\}_{i,j \in [n], i<j}$ 是两两独立的。

然而,这组事件并非相互独立。要证明这一点,我们只需找到一个反例。

考虑所有事件的交集 $\bigcap_{i,j \in [n], i<j} A_{i,j}$。这个事件表示“班上所有 $n$ 个人的生日都在同一天”。它的概率为:
\[
\mathbb{P}\left(\bigcap_{i,j \in [n], i<j} A_{i,j}\right) = \frac{m}{m^n} = \frac{1}{m^{n-1}}.
\]

另一方面,如果我们假设所有事件相互独立,那么它们的联合概率应该等于各自概率的乘积:
\[
\prod_{i,j \in [n], i<j} \mathbb{P}(A_{i,j}) = \left(\frac{1}{m}\right)^{\binom{n}{2}} = \frac{1}{m^{\binom{n}{2}}}.
\]

由于当 $n \ge 3$ 时,$\binom{n}{2} = \frac{n(n-1)}{2} > n-1$,所以
\[
\frac{1}{m^{\binom{n}{2}}} \ne \frac{1}{m^{n-1}}.
\]

因此,事件族 $\{A_{i,j}\}_{i,j \in [n], i<j}$ 不是相互独立的。

现在我们来解决最经典的问题:取 $m=30$,请问至少需要多少人 ($n$),才能使“班上存在两人同生日”的概率大于 $\frac{1}{2}$?

记事件 $B = \bigcup_{i,j \in [n], i<j} A_{i,j}$,即“至少有两人同生日”。我们通常用其补事件“所有人生日都不同”来计算:
\[
\mathbb{P}(B) = 1 - \mathbb{P}(B^c) = 1 - \mathbb{P}\left(\bigcap_{i,j \in [n], i<j} A_{i,j}^c\right).
\]

根据排列组合,所有人生日都不同的概率为:
\[
\mathbb{P}(B^c) = \frac{m \cdot (m-1) \cdot (m-2) \cdots (m-n+1)}{m^n} = \frac{m!}{(m-n)! \cdot m^n}.
\]

因此,
\[
\mathbb{P}(B) = 1 - \frac{m!}{(m-n)! \cdot m^n}.
\]

我们代入 $m=30$ 进行计算:
- 当 $n=6$ 时,$\mathbb{P}(B) \approx 1 - \frac{30 \cdot 29 \cdot 28 \cdot 27 \cdot 26 \cdot 25}{30^6} \approx 0.4136$。
- 当 $n=7$ 时,$\mathbb{P}(B) \approx 1 - \frac{30 \cdot 29 \cdot 28 \cdot 27 \cdot 26 \cdot 25 \cdot 24}{30^7} \approx 0.5308$。

因此,当 $n$ 至少为 $7$ 时,班上存在两人同生日的概率才大于 $\frac{1}{2}$。

这个结果再次印证了“生日悖论”的惊人之处:即使在只有30天的“年份”里,也只需要7个人,就有超过一半的概率出现生日重复!


\begin{example}[秘书问题]
某公司需要招聘一名秘书,共有 $n$ 位应聘者。为了节省时间和经济成本,公司决定采用一种“最优停止”策略:首先面试前 $K$ 位应聘者,仅记录他们的表现作为参考标准,但不录用任何人;从第 $K+1$ 位应聘者开始,只要遇到一位比之前所有面试者都优秀的候选人,就立即录用此人,后续的应聘者不再考虑;如果直到最后都没有找到这样的人选,则招聘失败。

这个问题的结论常常令人拍案叫绝。直觉上,我们可能会认为应该在中间某个位置开始选择,或者干脆直接选最后一个。但通过严谨的概率计算,我们会发现存在一个最优的“观察期” $K$,使得成功选到最佳人选的概率最大化。

\end{example}

令事件 $A$ 表示公司最终成功招聘到 $n$ 人中的最佳人选。对于每个 $j \in [n]$,令事件 $B_j$ 表示最佳人选恰好排在第 $j$ 个面试的位置。

我们需要计算条件概率 $\mathbb{P}(A|B_j)$:如果 $j \le K$,即最佳人选出现在前 $K$ 位面试者中,那么根据策略,公司不会录用他/她,因此 $\mathbb{P}(A|B_j) = 0$;如果 $j > K$,即最佳人选出现在第 $K+1$ 位或之后。此时,事件 $A$ 发生当且仅当在前 $j-1$ 位面试者中,实力最强的人恰好出现在前 $K$ 位之中。因为只有这样,第 $j$ 位的最佳人选才会被识别为“比之前所有人都好”。由于前 $j-1$ 位面试者的实力排序是随机的,最强者等可能地出现在这 $j-1$ 个位置中的任何一个,因此其落在前 $K$ 个位置的概率是 $\frac{K}{j-1}$。

综上所述:
\[
\mathbb{P}(A|B_j) =
\begin{cases}
0, & \text{若 } j \le K, \\
\frac{K}{j-1}, & \text{若 } j > K.
\end{cases}
\]

根据全概率公式,我们可以计算出总体的成功概率 $\mathbb{P}(A)$:
\[
\mathbb{P}(A) = \sum_{j=1}^{n} \mathbb{P}(A \cap B_j) = \sum_{j=1}^{n} \mathbb{P}(A|B_j) \cdot \mathbb{P}(B_j).
\]

由于 $\mathbb{P}(B_j) = \frac{1}{n}$ 对于所有 $j$ 都成立,代入上式得:
\[
\mathbb{P}(A) = \sum_{j=K+1}^{n} \frac{K}{j-1} \cdot \frac{1}{n} = \frac{K}{n} \sum_{j=K+1}^{n} \frac{1}{j-1} = \frac{K}{n} \sum_{j=K}^{n-1} \frac{1}{j}.
\]

当 $n$ 和 $K$ 都足够大时,我们可以用积分来近似求和:
\[
\sum_{j=K}^{n-1} \frac{1}{j} \approx \int_{K}^{n} \frac{1}{x} dx = \ln n - \ln K = \ln \frac{n}{K}.
\]

因此,
\[
\mathbb{P}(A) \approx \frac{K}{n} \ln \frac{n}{K} = -\frac{K}{n} \ln \frac{K}{n}.
\]

令 $x = \frac{K}{n}$,则 $\mathbb{P}(A) \approx -x \ln x$。这是一个经典的优化问题,函数 $f(x) = -x \ln x$ 在区间 $(0,1)$ 上的最大值出现在 $x = \frac{1}{e}$ 处,最大值为 $\frac{1}{e} \approx 0.3679$。

因此,最优的策略是设置 $K \approx \frac{n}{e}$。在实际操作中,我们应取最接近 $\frac{n}{e}$ 的整数,即 $K = \left\lfloor \frac{n}{e} \right\rfloor$ 或 $K = \left\lceil \frac{n}{e} \right\rceil$。

此时,成功选到最佳人选的概率约为 $36.79\%$。这个结果非常惊人——即使有成百上千的候选人,我们只需观察大约前 $37\%$ 的人,然后选择后面第一个比他们都优秀的人,就能以超过三分之一的概率获得最佳人选!


\begin{example}[无限悖论]
假设我们有可数无穷个球,用 $k = 1, 2, \dots$ 来编号,并且有一个无限大的箱子。我们考察一个“放球”与“拿球”的过程。首先是放球:在12点前1分钟,我们把 $1,2,\dots,10$ 号球放进去;在12点前 $\frac{1}{2}$ 分钟,我们把 $11, 12, \dots, 20$ 号球放进去;在12点前 $\frac{1}{4}$ 分钟,我们把 $21, 22, \dots, 30$ 号球放进去,以此类推。对于 $n = 0, 1, 2, \dots$,我们在12点前 $2^{-n}$ 分钟把 $10n+1, 10n+2, \dots, 10(n+1)$ 号球放进去。

然后我们再使用不同的方式把球拿出来。

\begin{leftbarquote}
    \quad \quad 对于 $n = 0, 1, 2, \dots$,我们在12点前 $2^{-n}$ 分钟放完球后,把第 $10(n+1)$ 号球拿出来。
\end{leftbarquote}

我们假设放球和拿球都是瞬时完成的,我们来计算一下,在12点的时候,箱子里有多少个球。显然,这种情况箱子里会有无穷多个球,因为所有编号不是10的倍数的球都被放了进去并且没有被拿出来。

我们换另外一种拿球的方式。

\begin{leftbarquote}
    \quad \quad 对于 $n = 0, 1, 2, \dots$,我们在12点前 $2^{-n}$ 分钟放完球后,把第 $(n+1)$ 号球拿出来。
\end{leftbarquote}

使用这种拿球的策略,在12点的时候,箱子里有多少个球呢?一番思考之后,不难发现,箱子里一个球都没有!因为对于每一个 $k \in \mathbb{N}$,第 $k$ 号球在 $n = k - 1$ 的时候被拿出来了。

和第一种情况一样,每次都是只拿了一个球出来,居然产生的结果截然不同。我现在想问,如果我随机的选一个球出来,又当如何呢?

\begin{leftbarquote}
    \quad \quad 对于 $n = 0, 1, 2, \dots$,我们在12点前 $2^{-n}$ 分钟放完球后,我们从箱子里均匀随机的拿一个球出来。
\end{leftbarquote}

我们来计算每一个球在12点的时候留在箱子里的概率。我们这儿以1号球为例,其它的球的计算方式类似。对于每一个 $n = 0, 1, 2, \dots$,我们用事件 $A_n$ 来表示在第 $n$ 轮操作之后(即12点前 $2^{-n}$ 分钟放球拿球的操作完成之后),1号球还在箱子里这个事件。我们关注的是事件
\[
A_\infty := \bigcap_{n \geq 0} A_n = \lim_{n \to \infty} A_n
\]

发生的概率。根据我们上一节课介绍的概率测度的连续性,我们有 $\mathbb{P}(\lim_{n \to \infty} A_n) = \lim_{n \to \infty} \mathbb{P}(A_n)$。因此,我们只需要计算 $\mathbb{P}(A_n)$,第 $n$ 轮结束后1号球在箱子里的概率。对于 $n = 0, 1, \dots$,我们再定义一个事件 $B_n$,用来表示,在第 $n$ 轮的时候拿出的球不是1号球,则有 $A_n = B_0 \cap B_1 \cap \cdots \cap B_n$。

使用条件概率的链式法则,我们可以得到
\[
\mathbb{P}(A_n) = \mathbb{P}\left( \bigcap_{i=1}^n B_i \right) = \prod_{k=0}^n \mathbb{P}\left( B_k \,\middle|\, \bigcap_{i < k} B_i \right).
\]

我们把 $\mathbb{P}(A_n)$ 写成这样的目的是因为对于每一个 $k = 0, 1, \dots, n$,条件概率 $\mathbb{P}(B_k \mid \bigcap_{i<k} B_i)$ 有着自然的组合意义,并非常好计算:在 $\bigcap_{i<k} B_i$ 的条件下,这对应于箱子里有 $9(k+1) + 1$ 个球,其中一个1号球,我们拿了一个球,但它不是1号球的概率。显然,这个概率是 $1 - \frac{1}{9(k+1)+1}$。所以,我们有
\[
\mathbb{P}(A_n) = \prod_{k=0}^n \left( 1 - \frac{1}{9(k+1)+1} \right) \leq e^{-\sum_{k=0}^n \frac{1}{9k+10}}.
\]

我们知道级数 $\sum_{k=0}^n \frac{1}{9k+10}$ 是发散的,因此, $\lim_{n \to \infty} \mathbb{P}(A_n) = 0$。

所以,我们知道了,在12点的时候,1号球在箱子里的概率是0。我们用 $S_i$ 来表示第 $i$ 号球在12点的时候还在箱子里的概率。我们刚才计算得知 $\mathbb{P}(S_1) = 0$。可以使用类似的方法,我们能够得到对于任意 $n \in \mathbb{N}$,$\mathbb{P}(S_n) = 0$。我们现在关心的是在12点的时候箱子里有球的概率,即 $\mathbb{P}(\exists n \in \mathbb{N}, S_n)$。我们使用上节课讲过的 union-bound,可以得到
\[
\mathbb{P}(\exists n \in \mathbb{N}, S_n) \leq \sum_{n \geq 1} \mathbb{P}(S_n) = 0.
\]
\end{example}

\section{Example:二分图完美匹配的概率与容斥原理}

在本节中,我们将探讨一个有趣的概率问题:在一个给定的二分图上,随机选取一个映射,它恰好构成一个完美匹配的概率是多少?考虑一个无向且连通的二分图 $G = (V, E)$,其中顶点集被划分为两个大小相等的部分:$V = V_1 \cup V_2$,满足 $|V_1| = |V_2| = n$ 且 $V_1 \cap V_2 = \varnothing$。我们假设 $V_1$ 和 $V_2$ 内部均无边相连。

我们可以用一个 $n \times n$ 的邻接矩阵 $A \in \{0,1\}^{n \times n}$ 来描述这个图。矩阵元素 $A_{i,j}$ 定义如下:
\[
A_{i,j} =
\begin{cases}
1, & \text{如果 } V_1 \text{ 中的第 } i \text{ 个点与 } V_2 \text{ 中的第 } j \text{ 个点之间有边}, \\
0, & \text{否则}.
\end{cases}
\]

我们的目标是:从所有可能的映射 $f: [n] \to [n]$ 中均匀随机地选取一个,计算它恰好对应图 $G$ 的一个完美匹配的概率。

这里,样本空间 $\Omega$ 是所有映射的集合,记为 $T_n = \{ f: [n] \to [n] \mid \forall i \in [n], A_{i,f(i)} = 1 \}$。注意,这里的映射 $f$ 不一定是置换,它只要求对于每个 $i$,其像 $f(i)$ 在 $V_2$ 中对应的点必须与 $i$ 相连。我们关心的事件 $M$ 是“所选映射 $f$ 正好是一个完美匹配”,即 $f$ 必须是一个置换函数,并且对所有的 $i$ 都有 $A_{i,f(i)} = 1$。

首先,我们直接写出事件 $M$ 的概率。由于概率测度 $\mathbb{P}$ 是均匀分布,事件 $M$ 的概率等于完美匹配的数量除以总的映射数量。令 $S_n$ 表示 $[n]$ 上所有置换函数的集合。一个置换 $\sigma \in S_n$ 构成完美匹配当且仅当 $\prod_{i=1}^{n} A_{i,\sigma(i)} = 1$。因此,完美匹配的总数就是 $\sum_{\sigma \in S_n} \prod_{i=1}^{n} A_{i,\sigma(i)}$。而样本空间 $T_n$ 的大小是 $\prod_{i=1}^{n} \left( \sum_{j \in [n]} A_{i,j} \right)$,因为对于每个 $i \in [n]$,都有 $\sum_{j \in [n]} A_{i,j}$ 种选择 $f(i)$ 的方式。

所以,我们得到第一个直观的概率表达式:
\[
\mathbb{P}(M) = \frac{\sum_{\sigma \in S_n} \prod_{i=1}^{n} A_{i,\sigma(i)}}{|T_n|}.
\]

这个表达式虽然清晰,但计算起来却非常困难,因为它需要枚举所有 $n!$ 个置换。为了找到一个更高效的计算方法,我们引入容斥原理。这是解决“至少有一个”或“恰好没有”这类计数问题的核心工具。我们定义一系列事件来刻画“映射的值域”。对于每个 $i \in [n]$,令事件 $E_i = \{ f \in T_n \mid f^{-1}(i) = \varnothing \}$,表示“$V_2$ 中的第 $i$ 个点没有原像”。

对于任意子集 $J \subseteq [n]$,我们定义:
\begin{itemize}
\item 事件 $E_J = \{ f \in T_n \mid \forall j \in J, f \in E_j; \forall j \notin J, f \notin E_j \}$,表示“映射的值域恰好是 $[n] \setminus J$”。
\item 事件 $E_J^+ = \{ f \in T_n \mid \forall j \in J, f \in E_j \}$,表示“映射的值域是 $[n] \setminus J$ 的子集”。
\end{itemize}

显然,事件 $E_\varnothing$ 就是我们关心的“映射的值域是整个 $[n]$”,即映射是一个满射。而在我们的问题中,由于 $V_1$ 和 $V_2$ 大小相同,一个满射映射就是一个双射,也就是一个置换。因此,事件 $E_\varnothing$ 等价于事件 $M$。

根据容斥原理,我们有:
\[
\mathbb{P}(M) = \mathbb{P}(E_\varnothing) = \sum_{J \subseteq [n]} (-1)^{|J|} \mathbb{P}(E_J^+).
\]

这个公式的证明可以通过分析每个映射 $f$ 在等式两边被计算的次数来完成。对于一个属于 $E_I$ 的映射 $f$,它在左边只被计算一次;在右边,它会被所有满足 $I \subseteq J$ 的项 $(-1)^{|J|} \mathbb{P}(E_J^+)$ 计算,其总贡献为 $\sum_{J \supseteq I} (-1)^{|J|} = (-1)^{|I|} \sum_{k=0}^{n-|I|} \binom{n-|I|}{k} (-1)^k = 0$(当 $I \neq \varnothing$ 时),只有当 $I = \varnothing$ 时,其贡献为1。这证明了公式的正确性。

现在,我们只需要计算 $\mathbb{P}(E_J^+)$ 即可。事件 $E_J^+$ 表示映射的值域不包含 $J$ 中的任何点,即所有 $f(i)$ 都必须落在 $[n] \setminus J$ 中。这意味着,对于每一个 $i \in [n]$,我们只能从 $V_2$ 中那些与 $i$ 相连且不属于 $J$ 的点中进行选择。因此,对于固定的 $i$,可选的点的数量是 $\sum_{j \in [n] \setminus J} A_{i,j}$。

由于每个 $i$ 的选择是独立的,事件 $E_J^+$ 包含的映射总数为:
\[
|E_J^+| = \prod_{i=1}^{n} \left( \sum_{j \in [n] \setminus J} A_{i,j} \right).
\]

于是,我们有:
\[
\mathbb{P}(E_J^+) = \frac{|E_J^+|}{|T_n|} = \frac{\prod_{i=1}^{n} \left( \sum_{j \in [n] \setminus J} A_{i,j} \right)}{|T_n|}.
\]

将此代入容斥原理的公式,我们得到了著名的 \textbf{Ryser公式}:
\[
\mathbb{P}(M) = \frac{\sum_{J \subseteq [n]} (-1)^{|J|} \prod_{i=1}^{n} \left( \sum_{j \in [n] \setminus J} A_{i,j} \right)}{|T_n|}.
\]

让我们比较一下两种方法的计算效率:
\begin{itemize}
\item 直接法需要计算分子 $\sum_{\sigma \in S_n} \prod_{i=1}^{n} A_{i,\sigma(i)}$ 需要遍历所有 $n!$ 个置换,时间复杂度为 $O(n \cdot n!)$;
\item 而\textbf{Ryser公式}只需要对 $2^n$ 个子集 $J$ 进行求和。对于每个 $J$,计算内层乘积需要 $O(n^2)$ 时间(因为需要对每个 $i$ 计算一个和)。因此,总的时间复杂度为 $O(n^2 \cdot 2^n)$。虽然 $O(n^2 \cdot 2^n)$ 仍然是指数级的,但它比 $O(n!)$ 要好得多。
\end{itemize}

更重要的是,在计算机科学的理论框架下,如果假设“指数时间假设”(Exponential Time Hypothesis, ETH)成立——即不存在能在 $2^{O(n)}$ 时间内解决 $n$ 个变量布尔公式的算法——那么Ryser公式在渐进意义上就是计算这个概率的最优算法。

\section{Example: Karger的最小割算法}
我们来看一个经典的算法问题,求图上的最小割。给定一个连通无向图 $G(V, E)$,我们说一个边集 $C \subseteq E$ 是一个割,当且仅当把 $C$ 从图中删掉之后,得到的图 $G(V, E \setminus C)$ 是不连通的。最小割问题即寻找图上最小的一个割。

学过算法的同学都知道,我们可以用最大流的方法来寻找最小割:固定一个源 $s$ 并枚举所有可能的汇 $t$,对每一个源-汇对 $(s, t)$ 求解最大流,其残量网络(residual network)给出了最小割。使用2022年最快的\href{https://en.wikipedia.org/wiki/Maximum_flow_problem}{\underline{最大流算法}},我们需要 $nm^{1+o(1)}$ 的时间来寻找这个最小割,其中 $n = |V|$, $m = |E|$。我们来介绍一个随机算法,实现非常简单,并且在有效的优化之后比基于最大流的算法要更快。

我们定义图上的所谓\textbf{缩边(contraction)}的操作。给定边 $e = \{u, v\} \in E$,我们缩掉 $e$ 的操作指的是把 $u$ 和 $v$ 合并成一个点,并且删掉之前所有 $u$ 与 $v$ 之间的边(他们俩与别的点相连的边依然保留)。我们把缩完之后的图记作 $G/e$。如下图所示,我们缩掉了边 $\{1, 2\}$。

\begin{figure}[h]
\centering
\includegraphics[width=0.7\textwidth]{figure/002.png}
\label{fig:karger}
\end{figure}

Karger算法非常简单,从 $G$ 出发,每一次随机选择一条边,把它缩掉。重复执行 $n - 2$ 次之后,图中只剩两个点了,然后输出所有剩下的边。

\begin{algorithm}
\While{G contains more than two vertices}{
    Choose an edge $e$ uniformly at random\;
    Contract $e$ in $G$\;
}
\Return{remaining edges}\;
\end{algorithm}

这个算法“有可能”成功的原因也很简单,由于我们关心的是“最小的”割,那么算法的每一步,选到割中的边的概率都不会太大。当然了,要谈论这个概率,我们需要有合适的概率空间。一个自然的样本空间可以是算法执行过程中所有(有序的)删除的边的序列。我们可以在上面定义合适的概率测度。

我们设 $C$ 是一个固定的最小割,并且其大小是 $k$。我们来计算算法最终输出 $C$ 的概率。很容易发现,算法最终输出 $C$,当且仅当算法执行 while 的循环中,每一次都没有选到 $C$ 中的边。因此,我们用 $A_k$ 来表示上面代码里面第 $k$ 次执行完 while 循环之后,还没有删过任何一个 $C$ 边的这个事件。为了分析 $\mathbb{P}(A_k)$,我们对于每一个 $k = 1, 2, \dots, n-2$,定义事件 $B_k$ 为“第 $k$ 次 while 循环选择的不是 $C$ 中边”这一事件。那么显然有 $A_k = \bigcap_{i=1}^k B_i$。因此,由条件概率的链式法则,
\[
\mathbb{P}(\text{算法输出 } C) = \mathbb{P}(A_{n-2}) = \prod_{i=1}^{n-2} \mathbb{P}\left( B_i \,\middle|\, \bigcap_{j=1}^{i-1} B_j \right).
\]

我们希望给上述概率一个下界,即每一轮均有比较大的概率不选到 $C$ 中的边。由于我们每一轮是均匀的选,因此我们只需要证明,对于第 $i$ 轮,在已知 $\bigcap_{j=1}^{i-1} B_j$,即前 $i-1$ 轮均没有选到 $C$ 中边的情况下,图中剩余的边足够多即可。一个最重要的观察是 > 此时,图中每一个顶点的度数均不少于 $k$。

这件事情成立的原因是,如果有一个顶点的度数 $< k$,那么在当前的图中,这个顶点与其邻居相连的边,构成了该图中的一个割。同时,也不难发现,这个割也一定是原图 $G$ 中的一个割(因为收缩这种操作不会破坏它是割的性质)。但这就说明,我们找到了一个大小比 $k$ 更小的割,这与我们假设 $C$ 是最小的割矛盾。

有了这个观察,因为我们知道算法在第 $i-1$ 轮后,还剩余 $n - i + 1$ 个顶点。所以第 $i$ 轮开始的时候,图中至少还有 $\frac{k}{2} \cdot (n - i + 1)$ 条边。于是,我们有
\[
\mathbb{P}\left( B_i \,\middle|\, \bigcap_{j=1}^{i-1} B_j \right) \geq 1 - \frac{2k}{k(n - i + 1)} = \frac{n - i - 1}{n - i + 1}.
\]

这说明,
\[
\mathbb{P}(A_{n-2}) \geq \prod_{i=1}^{n-2} \frac{n - i - 1}{n - i + 1} = \frac{2}{n(n - 1)}.
\]

因此,我们的算法有 $\frac{2}{n(n-1)}$ 的概率能够输出最小割 $C$。我们可以重复这个算法 $N = 50n(n - 1)$ 次,并且输出这 $N$ 次中找到最小的那个割。那么,这个割是最小割的概率至少有
\[
1 - \left(1 - \frac{2}{n(n - 1)}\right)^N \geq 1 - e^{-100}.
\]

最后我们来计算一下算法的复杂性。我们要重复算法 $N = O(n^2)$ 次,每一次要进行 $n - 2$ 次收缩操作。如果我们使用\href{https://en.wikipedia.org/wiki/Disjoint-set_data_structure}{\underline{并查集}}来维护的话,大约需要一共需要 $\tilde{O}(n^2 m)$ 的时间。这个时间是比前文提到的使用最大流的算法要慢的(都怪最大流算法发展的太快了,上一次讲的时候最大流还要 $\tilde{O}(nm)$ 呢),但好处在于实现非常简单。事实上,Karger算法可以进一步改进,使得只需要 $\tilde{O}(n^2)$ 的时间即可。下面我们便来简要说明这一点。

Karger算法的一个主要瓶颈在于,它需要将图一直收缩到只剩两个顶点。如果我们能在图规模还比较大的时候就停止收缩,转而用其他方法解决子问题,是否可以更快呢?答案是肯定的。这个改进的算法被称为 Karger-Stein 算法,其核心思想是:当图被收缩到一定规模时,不再继续收缩,而是递归地应用算法本身来解决子问题。

具体来说,我们定义一个函数 $\texttt{KS}(G)$,其工作流程如下:

\begin{algorithm}[H]
\SetAlgoLined
\KwIn{一个无向图 $G = (V, E)$}
\KwOut{图 $G$ 的最小割}
\eIf{$|V| \le 6$}{
    使用暴力枚举法直接求出最小割\;
}{
    设 $t = \left\lceil 1 + \frac{|V|}{\sqrt{2}} \right\rceil$\;
    $G_1 \leftarrow \texttt{contract}(G, t)$ \tcp*{将 $G$ 收缩至 $t$ 个顶点}
    $G_2 \leftarrow \texttt{contract}(G, t)$ \tcp*{独立地再做一次}
    \Return{ $\min( \texttt{KS}(G_1), \texttt{KS}(G_2) )$ }\;
}
\end{algorithm}

这个算法的巧妙之处在于,它通过递归调用自身,将一个大问题分解为两个规模约为原图 $\frac{1}{\sqrt{2}}$ 倍的子问题。我们需要证明,这个递归算法以较高的概率能返回正确的最小割。

首先,我们分析 $\texttt{contract}(G, t)$ 这一步骤的成功概率。令 $p(G, t)$ 表示在将图 $G$ 收缩至 $t$ 个顶点的过程中,从未碰到最小割 $C$ 中任何边的概率。根据我们在课堂上推导的结论,当 $t = \left\lceil 1 + \frac{n}{\sqrt{2}} \right\rceil$ 时,有:
\[
p(G, t) = \mathbb{P}(A_{n-t}) \geq \prod_{i=1}^{n-t} \frac{n-i-1}{n-i+1} = \frac{t(t-1)}{n(n-1)} > \frac{1}{2}.
\]

也就是说,单次收缩操作有超过一半的概率保留了最小割 $C$。

接下来,我们假设存在一个子程序 $\texttt{KS}$,对于规模为 $t$ 的图 $G'$,它能以概率 $p(t) \geq \frac{c}{\log t}$ 输出正确的最小割(其中 $c$ 是一个常数)。现在,我们从原图 $G$ 出发,独立运行两次 $\texttt{contract}$ 得到 $G_1$ 和 $G_2$,然后分别对它们调用 $\texttt{KS}$。我们关心的是,最终输出的两个结果中,较小的那个是原图最小割的概率。

记事件 $B$ 为“$\texttt{KS}(G_1)$ 和 $\texttt{KS}(G_2)$ 中较小的那个是原图 $G$ 的最小割”。记事件 $B_1$ 为“$\texttt{KS}(G_1)$ 是 $G_1$ 的最小割”,事件 $B_2$ 为“$\texttt{KS}(G_2)$ 是 $G_2$ 的最小割”。

根据全概率公式,我们可以计算:
\[
\mathbb{P}(B) = 1 - \mathbb{P}(B^c) = 1 - \left[ \mathbb{P}(A_{n-t}^c) + \mathbb{P}(A_{n-t})\mathbb{P}(B_1^c) \right] \cdot \left[ \mathbb{P}(A_{n-t}^c) + \mathbb{P}(A_{n-t})\mathbb{P}(B_2^c) \right].
\]

代入已知的概率下界:
\[
\mathbb{P}(B) > 1 - \left(1 - \frac{1}{2} \cdot \frac{c}{\log t} \right)^2 \geq \frac{c}{\log t} - \frac{c^2}{4 \log^2 t} = \Omega\left(\frac{1}{\log n}\right).
\]

这表明,即使在递归调用中,算法依然能保持一个与 $\frac{1}{\log n}$ 同阶的成功概率。

最后,我们分析算法的时间复杂度。设 $T(n)$ 为算法在处理 $n$ 个顶点的图时所需的期望时间。

根据算法结构,我们有:
\[
T(n) = 2T\left(\left\lceil 1 + \frac{n}{\sqrt{2}} \right\rceil\right) + O(n^2).
\]

这是一个标准的分治递推式。我们可以使用主定理(Master Theorem)来求解。这里,子问题规模缩小因子为 $a = 2$,规模缩减比例为 $b = \sqrt{2}$,非递归部分为 $f(n) = O(n^2)$。由于 $\log_b a = \log_{\sqrt{2}} 2 = 2$,且 $f(n) = \Theta(n^{\log_b a})$,根据主定理的情况二,我们得到:
\[
T(n) = \Theta(n^2 \log n).
\]

然而,这还不是最终答案。因为我们只需要以常数概率(例如 $\frac{2}{3}$)成功即可,所以我们可以将整个算法独立重复 $O(\log n)$ 次,取其中最好的结果。这样,总的时间复杂度变为:
\[
O(\log n) \cdot T(n) = O(n^2 \log^2 n) = \tilde{O}(n^2).
\]

\newpage