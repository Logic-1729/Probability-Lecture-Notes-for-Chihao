\chapter{随机变量的特征函数}

\section{特征函数的定义以及基本性质}

我们之前介绍过随机变量 $X$ 的矩生成函数 $M_X(t) = \E{e^{tX}}$。它是研究随机变量的矩的有力工具。我们证明过的一个重要的性质是如果 $M_X(t)$ 在 $t=0$ 附近的一个邻域内存在的话,那么 $X$ 的任意一阶矩都存在,并且对于 $k\in \bb N$,我们有
\[
    \E{X^k} = \dv[k]{}{t} M_X(0).
\]
这个结论同时告诉我们,如果存在某个 $m\in \bb N$,使得 $X^m$ 是不可积的,那么 $M_X(t)$ 就不存在。但往往我们还是会对 $k<m$ 时候 $X$ 的 $k$ 阶矩感兴趣,矩生成函数就无能为力了。这个时候就要小修改一下矩生成函数的定义。我们定义 $X$ 的特征函数
\[
    \varphi_X(t) \defeq \E{e^{i t X}},
\]
这儿 $i=\sqrt{-1}$ 是虚数单位。注意到特征函数和矩生成函数唯一的不同是把 $t$ 换成了 $it$。在 $X$ 有概率密度 $f(x)$ 的时候,它们分别对应了对 $f(x)$ 的\href{https://en.wikipedia.org/wiki/Fourier_transform}{\underline{傅里叶变换}}和\href{https://en.wikipedia.org/wiki/Laplace_transform}{\underline{拉普拉斯变换}}。我们马上可以看到,$\varphi_X(t)$ 总是存在的。

我们先来研究一下特征函数的一些基本性质。

\begin{enumerate}
\item 根据欧拉公式
    \[
        \varphi_X(t) = \E{\cos tX} + i\E{\sin tX}.
    \]
    
    由于 $\sin$ 和 $\cos$ 都是有界函数,因此可以看出 $\varphi_X(t)$ 对于任意 $t$ 都是存在的。这也是和 $M_X(t)$ 的一个本质区别。
    \item 如果 $X$ 存在密度函数 $f_X(x)$,那么
    \[
        \varphi_X(t) = \int_{-\infty}^\infty f_X(x)\cdot e^{itx} \d x
    \]
    
    是 $f_X(x)$ 的傅立叶变换。根据\href{https://en.wikipedia.org/wiki/Fourier_inversion_theorem}{\underline{傅里叶逆变换定理}},
    \[
        f_X(x) = \lim_{T\to\infty} \int_{-T}^T \varphi_X(t)\cdot e^{-itx} \d t.
    \]
    
    注意到对于这个不定积分我们使用了\href{https://en.wikipedia.org/wiki/Cauchy_principal_value}{\underline{柯西主值}},因为 $\varphi_X(t)$ 不一定可积。这个结论告诉我们,密度函数唯一确定了特征函数,而特征函数也唯一确定了密度函数(也就是 $X$ 的分布)。

    \item 上述结论可以被推广到一般的随机变量(当 $X$ 不存在密度函数的情形),被称作 \href{https://proofwiki.org/wiki/L\%C3\%A9vy\%27s_Inversion_Formula}{\underline{Lévy 逆定理}}。设 $\bar F(x) = \frac{1}{2}\tp{F(x)+F(x-)}$,其中 $F(x-)\defeq \lim_{z\uparrow x} F(z)$。那么
    \[
        \forall a<b,\quad \bar F(b) - \bar F(a) = \lim_{T\to\infty} \int_{-T}^T \frac{i}{2\pi t}\tp{e^{-ibt}-e^{-iat}}\cdot \varphi_X(t) \d t.
    \]

      我们不会证明这个结论(证明可以查看上面链接),但强调一下 Lévy 逆定理说明了随机变量的分布和其特征函数是相互唯一对应的(当然了,如果两个分布只在一个零测集上不一样,我们也认为它们相同)。
      
    \item 根据定义容易验证,如果两个随机变量 $X$ 和 $Y$ 独立,那么 $\varphi_{X+Y}(t) = \varphi_X(t)\cdot \varphi_Y(t)$。类似的结论可以推广到 $n$ 个相互独立的随机变量 $X_1,\dots,X_n$:$\varphi_{\sum_{i=1}^n X_i}(t) = \prod_{i=1}^n \varphi_{X_i}(t)$。
\end{enumerate}
\subsection{联合分布的特征函数}

对于定义在同一个概率空间上的两个随机变量 $X$ 和 $Y$,我们可以定义它们的联合特征函数
\[
\varphi_{(X,Y)}(s,t) = \E{e^{i\tp{sX+tY}}}.
\]

这个定义可以推广到任意 $n$ 个随机变量 $\*X = (X_1,\dots,X_n)$。它们的联合特征函数是
\[
\varphi_{\*X}\colon \*t\in \bb R^n \mapsto \E{e^{i\*t^\top \*X}}.
\]

Lévy逆定理可以被\href{https://en.wikipedia.org/wiki/Characteristic_function_(probability_theory)\#Inversion_formula}{\underline{推广}}到 $\*X$ 这样的随机向量的场合。类似的,联合特征函数也唯一(up to 零测集)确定了联合分布。

\section{特殊分布的特征函数计算}

我们来给那几位老伙计算算特征函数。

\begin{example}[$X\sim\!{Ber}(p)$]
显然有 $\varphi_X(t) = \E{e^{itX}} = 1-p+pe^{it}$.
\end{example}

\begin{example}[$X\sim\!{Binom}(n,p)$]

由于 $X$ 可以写成 $n$ 个分布为 $\!{Ber}(p)$ 的独立随机变量之和,根据我们前面提到的性质,$\varphi_X(t) = \tp{1-p+pe^{it}}^n$。
\end{example}

\begin{example}[$X\sim\!{Exp}(\lambda)$]

由于指数分布的概率密度是 $\forall x\ge 0,\;f_X(x) = \lambda e^{-\lambda x}$,所以
\[
    \varphi_X(t) = \E{e^{itX}} = \int_0^\infty \lambda e^{-\lambda x} e^{itx} \d x = \frac{\lambda}{\lambda-it}.
\]
\end{example}

\begin{example}[$X\sim\!{Pois}(\lambda)$]

泊松分布的概率质量函数是 $\forall n\in \bb N,\; p_X(n) = \frac{\lambda^n}{n!}e^{-\lambda}$,所以
\[
    \varphi_X(t) = \E{e^{itX}} = \sum_{n=0}^\infty \frac{\lambda^n}{n!}e^{-\lambda}e^{itn} = e^{-\lambda} \sum_{n=0}^\infty \frac{\tp{\lambda e^{it}}^n}{n!} = e^{-\lambda}e^{\lambda e^{it}}.
\]
\end{example}

\begin{example}[$X\sim\+N(0,1)$]

标准高斯分布的概率密度是 $\phi_X(x) = \frac{1}{\sqrt{2\pi}} e^{-\frac{x^2}{2}}$,所以
\[
    \varphi_X(t) = \E{e^{itX}} = \frac{1}{\sqrt{2\pi}} \int_{-\infty}^\infty e^{itx - \frac{x^2}{2}} \d x.
\]    
\end{example}

我们把积分里面的指数部分进行配方,然后做 $z=x-it$ 的换元,可以得到
\[
    \varphi_X(t) = \frac{e^{-\frac{t^2}{2}}}{\sqrt{2\pi}}\int_{\Im z = -t} e^{-\frac{1}{2}z^2} \d z.
\]

这是一个在复平面上的积分,我们的积分范围是 $\Im z = -t$ 的直线。对于给定的 $N>0$,我们计算如下图所示的围道积分。

\begin{figure}[h]
\centering
\includegraphics[width=0.5\textwidth]{figure/Figure25.1.png}
\end{figure}

我们把蓝色曲线记作 $L_N$,那么根据\href{https://en.wikipedia.org/wiki/Cauchy\%27s_integral_theorem}{\underline{柯西定理}}
\[
    \int_{L_N} e^{-\frac{1}{2}z^2}\d z = \int_{L_N^{(1)}} e^{-\frac{1}{2}z^2}\d z + \int_{L_N^{(2)}} e^{-\frac{1}{2}z^2}\d z + \int_{L_N^{(3)}} e^{-\frac{1}{2}z^2}\d z + \int_{L_N^{(4)}} e^{-\frac{1}{2}z^2}\d z = 0.
\]

注意到 $\lim_{N\to\infty} \int_{L_N^{(2)}} e^{-\frac{1}{2}z^2}\d z = \lim_{N\to\infty} \int_{L_N^{(4)}} e^{-\frac{1}{2}z^2}\d z = 0$,所以 
\[
\lim_{N\to\infty} \int_{L_N^{(1)}} e^{-\frac{1}{2}z^2}\d z = -\lim_{N\to\infty} \int_{L_N^{(3)}} e^{-\frac{1}{2}z^2}\d z = \int_{-\infty}^\infty e^{-\frac{1}{2}z^2} \d z = \sqrt{2\pi}.
\]

也就是说 $\int_{\Im z = -t} e^{-\frac{1}{2}z^2}\d z = \sqrt{2\pi}$. 代回 $\varphi_X(t)$ 的表达式我们便能得到
\[
    \varphi_X(t) = e^{-\frac{t^2}{2}}.
\]

Wow,居然和 $\phi_X(x)$ 的形式一致。高斯的傅里叶变换还是高斯。

\begin{example}[$X\sim\+N(\mu,\sigma^2)$]

如果 $\xi\sim\+N(0,1)$,那么 $X\defeq \sigma \xi+\mu\sim \+N(\mu,\sigma^2)$。根据定义,我们显然有
\[
    \varphi_X(t) = \E{e^{it(\sigma \xi+\mu)}} = e^{it\mu}\E{e^{it\sigma\xi}} = e^{it\mu}\varphi_\xi(\sigma t) = e^{-\frac{\sigma^2t^2}{2}+it\mu}.
\]
\end{example}
\begin{example}[$X\sim\+N\tp{\mu,\Sigma}$]

注意到这儿 $X=(X_1,\dots,X_n)$ 是一个 $n$ 维随机向量。我们计算它的联合特征函数。对于一个 $\*t = (t_1,\dots,t_n)^\top$,我们有
\[
    \varphi_X(\*t) = \E{e^{i\*t^\top X}}.
\]

我们注意到 $Y\defeq \*t^\top X$ 是一个高斯随机变量,我们只要计算出 $Y$ 的期望和方差,就能得到 $Y$ 的特征函数 $\varphi_Y(t)$。而 $\varphi_X(\*t) = \varphi_Y(1)$。

我们显然有 $\E{Y} = \*t^\top \E{X} = \*t^\top \mu$。我们知道 $X = A\xi+\mu$,其中 $\xi\sim\+N\tp{0,\!{Id}_n}$,矩阵 $A$ 满足 $AA^\top = \Sigma$。于是
\[
    \Var{Y} = \E{\tp{\*t^\top A\xi}^2} = \*t^\top A\E{\xi\xi^\top} A^\top\*t = \*t^\top\Sigma\*t.
\]

这便得到
\[
    \varphi_X(\*t) = \varphi_Y(1) = e^{-\frac{1}{2}\*t^\top\Sigma\*t +i\*t^\top \mu}.
\]
\end{example}
\subsection{多元高斯分布的刻画}

我们之前说一个 $n$-维随机变量 $X=(X_1,\dots,X_n)^\top$ 是多元高斯(或称高维高斯,联合高斯)的,当且仅当存在 $n\times n$ 维矩阵 $A$ 和 $n$ 维向量 $\mu$ 使得 $X = A\xi+\mu$,其中 $\xi$ 是一个每一维是独立 $\+N(0,1)$ 随机变量的 $n$ 维向量。并且我们知道 $X\sim \+N\tp{\mu,\Sigma}$,其中 $\Sigma = AA^\top$。我们现在给它一个新的刻画:

\begin{theorem}
$X=(X_1,\dots,X_n)^\top$ 是一个高维高斯向量当且仅当 $X_1,\dots,X_n$ 的任意线性组合是一个一维高斯随机变量。
\end{theorem}

定理的“仅当”方向是显然的,即如果 $X$ 是高维高斯,那么 $X_1,\dots,X_n$ 的任意线性组合也是高斯。这是由于每一个 $X_i$ 都可以写成 $\xi_1,\dots,\xi_n$ 的线性组合,于是任意 $X_1,\dots,X_n$ 的线性组合也可以写成 $\xi_1,\dots,\xi_n$ 的线性组合。而我们知道,独立高斯的线性组合依旧是高斯的。

我们现在来证明“当”。设 $\mu = \E{X}$,$\Sigma = \Cov{X}$。根据联合分布的 Lévy 逆定理,我们只要说明 $X$ 的(联合)特征函数是高维高斯的就行。也就是对于任何 $\*t\in \bb R^n$,计算 $\varphi_X(\*t) = \E{e^{i\*t^\top X}}$。根据条件,我们知道 $Y\defeq \*t^\top X$ 作为 $X_1,\dots,X_n$ 的一个线性组合是满足高斯分布的。因此我们有 $\varphi_X(\*t) = \varphi_Y(1)$。同样我们只需要计算 $Y$ 的期望和方差就能得到 $\varphi_Y$。
\[
\begin{aligned}
    \E{Y} &= \*t^\top \E{X} = \*t^\top \mu,\\
    \Var{Y} &= \E{\tp{\*t^\top X-\*t^\top\mu}^2} = \*t^\top\E{(X-\mu)(X-\mu)^\top}\*t = \*t^\top\Sigma\*t.
\end{aligned}
\]

这便说明 $\varphi_X(\*t) = \varphi_Y(1) = e^{-\frac{1}{2}\*t^\top\Sigma\*t +i\*t^\top \mu}$。也就是说 $X\sim \+N\tp{\mu,\Sigma}$。

\section{特征函数与随机变量的矩}

我们在一开始便提到过,我们对矩生成函数求导可以得到随机变量的矩。但这一操作的可行性需要随机变量的任意一阶矩均存在。如果 $X$ 对于 $m$ 阶矩不存在,但对于 $k<m$ 阶矩存在(回忆我们以前用 Jensen 不等式证明过对于 $p>1$,$X^p$ 可积可以蕴含 $X^{p-1}$ 可积),使用矩生成函数便不行了。下面这个结论,除了告诉我们可以使用特征函数来计算 up to $m-1$ 阶矩之外,对于研究特征函数本身的性质有着重要的意义。

\begin{theorem}
如果随机变量 $X$ 满足 $\E{\abs{X}^n}<\infty$,那么
\[
\varphi_X(t) = \sum_{k=0}^n \frac{(it)^k}{k!}\E{X^k} + o(\abs{t}^n).
\]

特别的,对于 $k=1,2,\dots,n$,$\dv[k]{}{t}\varphi_X(0) = i^k \E{X^k}$.
\end{theorem}

函数 $e^{itX}$ 的泰勒级数的前 $n$ 项是 $\sum_{k=0}^n \frac{1}{k!}X^k(it)^k$。因此,我们为了证明这个定理,需要讨论的事情主要是控制级数的余项,并据此说明可以交换求和和期望。

我们使用带有\href{https://en.wikipedia.org/wiki/Taylor\%27s_theorem\#Derivation_for_the_integral_form_of_the_remainder}{\underline{积分余项的泰勒级数}}:
\[
    f(z) = \sum_{k=0}^n \frac{f^{(k)}(0)}{k!}\cdot z^k  + \frac{1}{n!}\int_0^z (z-t)^n f^{(n+1)}(t) \d t.
\]

我们用 $R_n$ 表示把 $e^z$ 展开到 $n$ 次之后的余项,于是
\[
e^{ix} = \sum_{k=0}^n \frac{(ix)^k}{k!} + R_{n+1},\quad R_{n+1} = \frac{i^{n+1}}{n!}\int_0^x (x-t)^n e^{it}\d t.
\]

从这个表达式看起来,$\abs{R_{n+1}} \le \frac{\abs{x}^{n+1}}{(n+1)!}$。这在 $x\to 0$ 的时候是一个很好的上界,但是我们的条件是 $\E{\abs{X}^n}<\infty$,在 $x\to\infty$ 的时候 $\abs{x}^{n+1}$ 太大了。我们需要找一个 $x\to\infty$ 时更好的上界。注意到 
\[
R_{n+1} = R_n - \frac{(ix)^n}{n!} = \frac{i^n}{n!}\tp{n\int_0^x (x-t)^{n-1} e^{it}\d t - x^n}.
\]

于是 $\abs{R_{n+1}}\le \frac{2\abs{x}^n}{n!}$。我们便得到了
\[
    \abs{R_{n+1}}\le \frac{\abs{x}^{n+1}}{(n+1)!}\land \frac{2\abs{x}^n}{n!}.
\]

根据条件 $\E{\abs{X}^n}<\infty$,我们知道 
\[
    e^{itX} = \sum_{k=0}^n \frac{(itX)^k}{k!} + R_{n+1}'(X), \quad \abs{R_{n+1}'(X)} \le \frac{2\abs{t}^n\abs{X}^n}{n!}
\]

满足求和的每一项都是可积的。因此,根据期望的线性性,我们有
\[
    \E{e^{itX}} = \sum_{k=0}^n \frac{(it)^k}{k!}\E{X^k} +\E{R_{n+1}'(X)}.
\]

我们接着说明 $\E{\abs{R_{n+1}'(X)}} = o(\abs{t}^n)$ as $t\to 0$。这等价于 $\lim_{t\to 0} t^{-n}\E{\abs{R_{n+1}'(X)}} = 0$。根据我们上面的对于余项的上界可以知道 $t^{-n} \abs{R_{n+1}'(X)}\le \frac{2\abs{X}^n}{n!}$. 因此使用 DCT,
\[
    \lim_{t\to 0} t^{-n}\E{\abs{R_{n+1}'(X)}} = \E{\lim_{t\to 0}t^{-n}\abs{R_{n+1}'(X)}}\le \E{\lim_{t\to 0}\frac{t\abs{X}^{n+1}}{(n+1)!}} = 0.
\]

注意到我们在上面的分析中,同时用到了 $R_{n+1}$ 的两个上界,分别对应于 $x$ 很大和 $x$ 很小的时候。

\section{\href{https://en.wikipedia.org/wiki/L\%C3\%A9vy\%27s_continuity_theorem}{\underline{Lévy 连续性定理及应用}}}

特征函数的另一个重要性质说的是它与依分布收敛的关系。我们不加证明的给出结论

\begin{theorem}[Lévy 连续性定理]
给定一族(不一定定义在同一概率空间上的)随机变量 $X_1,X_2,\dots$。对于每一个 $n\ge 1$,我们用 $\varphi_n$ 表示 $X_n$ 的特征函数。如果 $\varphi_n(t)$ 逐点收敛到一个函数 $\varphi(t)$。那么下面两件事情等价。
\begin{itemize}
\item $\varphi(t)$ 在 $t=0$ 连续;
    \item 存在一个随机变量 $X$,它的特征函数是 $\varphi$,并且 $X_n\overset{D}{\to} X$。
\end{itemize}
\end{theorem}

另一方面,我们可以很容易验证,如果 $X_n\overset{D}{\to} X$,那么 $\varphi_n(t)$ 逐点收敛到 $\varphi(t)$ (类似我们之前用测试函数说明依分布收敛的证明,这儿对应的测试函数是 $h_t(x) = e^{itx}$ )。所以,我们知道,在 $\varphi(t)$ 在 $t=0$ 连续的条件下,依分布收敛和特征函数的逐点收敛是等价的。

我们简单说明一下,$\varphi(t)$ 在 $t=0$ 这一点连续的条件是必要的,否则 $\varphi(t)$ 有可能并不是任何函数的特征函数。比如说,我们让 $X_n\sim\+N(0,n)$,那么 $\varphi_n(t) = e^{-\frac{nt^2}{2}}$。它的极限是 $\varphi(t) = \bb I[t=0]$,在 $t=0$ 不连续,也不是任何随机变量的特征函数。

Lévy 连续定理是我们研究依分布收敛的重要工具。接下来我们看几个应用。

\subsection{泊松分布作为二项式分布的极限}

我们之前介绍泊松分布 $\!{Pois}(\lambda)$ 的时候是把它看成二项式分布 $\!{Binom}(n,p)$ 在 $np = \lambda$ 时候的极限。这个事实可以用特征函数一句话说明。注意到 $X\sim\!{Binom}(n,p)$ 的特征函数是
\[
    \tp{1-p+pe^{it}}^n = \tp{1-\frac{np-np e^{it}}{n}}^n = \tp{1-\frac{\lambda-\lambda e^{it}}{n}}^n\overset{n\to\infty}{\to} e^{-\lambda} e^{\lambda e^{it}}.
\]

而这正是 $\!{Pois}(\lambda)$ 的特征函数。

\subsection{中心极限定理的特征函数证明}

我们现在用特征函数来证明中心极限定理。

\begin{theorem}[中心极限定理]
如果独立同分布的随机变量 $X_1,X_2,\dots$ 满足 $\E{X_1}=\mu, \Var{X_1}=\sigma^2$ 均为有限的,那么
\[
\frac{S_n-n\mu}{\sigma\sqrt{n}}\overset{D}{\to} Y\sim\+N(0,1).
\]

\end{theorem}

我们不失一般性的假设 $\mu=0$, $\sigma^2=1$。我们已经证明了
\[
    \varphi_{X_1}(t) = \E{e^{itX_1}} = \E{1+itX_1-\frac{1}{2}t^2X_1^2+\eps(t^2)} = 1-\frac{1}{2}t^2+\eps(t^2).
\]

这儿 $\eps(t)$ 是一个满足 $\lim_{t\to 0} \frac{\eps(t)}{t} = 0$ 的函数。于是根据独立性
\[
  \varphi_{\frac{S_n}{\sqrt{n}}}(t)   = \varphi_{X_1}\tp{\frac{t}{\sqrt{n}}}^n = \tp{1-\frac{t^2}{2n}+\eps\tp{\frac{t^2}{n}}}^n\overset{n\to\infty}{\to} e^{-\frac{1}{2}t^2}.
\]

由 Lévy 连续性定理 $\frac{S_n}{\sqrt{n}}\overset{D}{\to}\xi\sim\+N(0,1)$。

可以看到,特征函数是处理独立随机变量和的有力工具。我们将在作业里尝试使用它去掉 CLT 里对于同分布的要求。