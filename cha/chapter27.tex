\chapter{离散鞅简介}

最后两次课,我们简单介绍一下离散\href{https://en.wikipedia.org/wiki/Martingale_(probability_theory)}{\underline{鞅论(martingale)}}。这是现代概率论的核心工具,有着丰富的内容和非凡的应用。因为课时原因,我们仅仅能够简单一瞥其芳容。

鞅的概念首先来自公平赌博游戏。假设你在玩一个猜大小的游戏。游戏的每一轮开始前,你可以“押大”或者“押小”。你可以押任意多的赌注,但无论如何,在每一轮中期望收益为零。因此,你的总财产的“期望”是保持不变。用数学的语言来描述这个性质,我们用 $X_t$ 表示第 $t$ 轮的收益,用 $Z_t$ 表示第 $t$ 轮结束之后的总财产。那么对于任何 $T>0$,$Z_T = Z_0+\sum_{t=1}^T X_t$。这个游戏是一个公平游戏指的是
\[
    \forall t\ge 0,\;\E{X_{t+1}\mid X_1,\dots,X_t} = 0,
\]

或者等价的
\[
     \forall t\ge 0,\;\E{Z_{t+1} \mid X_1,\dots,X_t} = Z_t.
\]

我们把这样一个性质抽象出来,便是一般的鞅的定义。注意到我们并不要求 $X_{t+1}$ 与 $X_1,\dots,X_t$ 独立,也就是你的赌注可以与之前的胜负有关,as long as 我们玩的是一个公平游戏就行。

\begin{itemize}

\item 设 $\set{X_n}_{n \geq 0}$ 和 $\set{Z_n}_{n \geq 0}$ 是两个随机变量序列。如果每一个 $Z_n$ 都是可积的,并且它们满足
    \[
    \forall n\ge 0,\;\E{Z_{n+1} \mid X_0, X_1, \dots, X_n} = Z_n,
    \]
    
    我们则称 $\set{Z_n}_{n \geq 0}$ 是相对于 $\set{X_n}_{n \geq 0}$ 的一个鞅。

    \item 有的时候,我们也会直接称一个序列 $\set{Z_n}_{n\ge 0}$ 是鞅。这个的意思是 $\set{Z_n}$ 相对于自己是鞅,即
    \[
    \forall n\ge 0,\;\E{Z_{n+1} \mid Z_0, Z_1, \dots, Z_n} = Z_n.
    \]
\end{itemize}


为了简化记号,接下来我们用 $\ol{X}_{i,j}$ 来表示随机变量 $(X_i, X_{i+1}, \dots, X_j)$。

我们注意到,条件期望 $\E{Z_{n+1} \mid \ol{X}_{0,n}}$ 实际上的定义是 $\E{Z_{n+1} \mid \sigma(\ol{X}_{0,n})}$。这便让我们可以用 $\sigma$-代数的语言更一般的定义鞅。

为此我们先引入滤链的概念。考虑概率空间 $(\Omega,\@F,\bb P)$。设 $\set{\@F_n}_{n \geq 0}$ 是一列 $\sigma$-代数。如果满足:
\[
\@F_0 \subseteq \@F_1 \subseteq \cdots \subseteq \@F_n \subseteq \@F_{n+1} \subseteq \cdots\subseteq \@F,
\]

则称 $\set{\@F_n}_{n \geq 0}$ 为一个\href{https://en.wikipedia.org/wiki/Filtration_(probability_theory)}{\underline{滤链(filtration)}}。直观上,它编码了逐渐增多的信息。

\begin{definition}[鞅]
设 $\set{Z_n}_{n \geq 0}$ 是一列可积的随机变量,并且对于每个 $n \geq 0$,$Z_n$ 是 $\@F_n$-可测的。如果
\[
\forall n\ge 0,\;\E{Z_{n+1} \mid \@F_n} = Z_n
\]

成立,则称 $\set{Z_n}_{n \geq 0}$ 是相对于 $\set{\@F_n}_{n \geq 0}$ 的鞅。
\end{definition}

这个定义直观上就是在说,基于前 $n$ 轮的信息,$Z_{n+1}$ 的平均值是保持不变的( $=Z_n$ )。

在上述定义中,如果把要求改为 $\forall n\ge 0,\;\E{Z_{n+1} \mid \@F_n} \leq Z_n$,则称 $\{Z_n\}_{n \geq 0}$ 是一个上鞅 (super-martingale)。类似地,如果 $\forall n\ge 0, \E{Z_{n+1} \mid \@F_n} \geq Z_n$,则称其为下鞅(sub-martingale)。

我们接下来看一些鞅的例子。

\begin{example}[一维随机游走]

考虑一个在整数集合 $\mathbb{Z}$ 上的随机游走,起点为 $0$。在每一步中,向左和向右移动的概率均为 $\frac{1}{2}$。设第 $n$ 步的移动由一个取值为 $\{-1, +1\}$ 的均匀随机变量 $X_n$ 表示。对于任何 $n\ge 1$,我们令
\[
S_n = S_{n-1}+X_n.
\]

容易可以验证 $\{S_n\}_{n \geq 0}$ 是相对于 $\set{X_n}_{n \geq 1}$ (或 $\{S_n\}_{n \geq 0}$ )的鞅:
\[
\forall n\ge 0,\; \E{S_{n+1} \mid \ol{X}_{0,n}} = \E{S_n + X_{n+1} \mid \ol{X}_{0,n}} = S_n + \E{X_{n+1} \mid \ol{X}_{0,n}} = S_n.
\]

更一般地,如果 $\E{X_{n+1} \mid \ol{X}_{0,n}} = \mu$,我们定义:
\[
\forall n\ge 1, Y_n = X_n - \mu, \; S'_n = S'_{n-1}+Y_n = S_n - n\cdot\mu.
\]

则 $\set{S'_n}_{n \geq 0}$ 是相对于 $\set{Y_n}_{n \geq 1}$ 的鞅。
\end{example}

\begin{example}[均值为 $1$ 的乘积]

考虑一个随机变量序列 $\{X_n\}_{n \geq 1}$,其中对于所有 $n \geq 1$,有 $\E{X_n \mid \ol{X}_{1,n-1}} = 1$。令:
\[
P_n = \prod_{k=1}^n X_k.
\]

我们可以验证 $\{P_n\}_{n \geq 0}$ 是相对于 $\{X_n\}_{n \geq 1}$ 的鞅:
\[
\E{P_{n+1} \mid \ol{X}_{1,n}} = \E{P_n \cdot X_{n+1} \mid \ol{X}_{1,n}} = P_n \cdot \E{X_{n+1} \mid \ol{X}_{1,n}} = P_n.
\]
\end{example}
\begin{example}[Galton-Watson 过程]

\href{https://en.wikipedia.org/wiki/Galton\%E2\%80\%93Watson_process}{\underline{Galton-Watson过程}}是一个用来建模和研究某一个贵族姓氏灭绝概率的模型。在这个模型里,用 $G_t$ 表示第 $t$ 代的个体数量(为了方便,只考虑男性)。假设所有个体彼此独立的繁殖,并且后代数量同分布。我们用 $X_{t,k}$ 表示第 $t$ 代第 $k$ 个个体的(男性)后代个数。设 $\mu = \E{X_{t,k}}$。于是我们有
\[
    G_{t+1} = \sum_{k=1}^{G_t} X_{t,k}.
\]

所以
\[
\E{G_{t+1} \mid G_t} = \E{\sum_{k=1}^{G_t} X_{t,k} \mid G_t} = G_t \cdot \E{X_{t,1}} = \mu\cdot G_t.
\]

定义 $M_t \defeq \mu^{-t} G_t$,则:
\[
\E{M_{t+1} \mid G_t} = \mu^{-(t+1)} \E{G_{t+1} \mid G_t} = \mu^{-t} G_t = M_t.
\]

因此,$\{M_t\}_{t \geq 1}$ 是相对于 $\{G_t\}_{t \geq 1}$ 的鞅。
\end{example}
\begin{example}[波利亚的罐子(Pólya’s Urn)]

假设一个罐子中有若干白球和黑球,除了颜色外,所有球完全相同。考虑以下随机过程:每轮随机取一个球,记录其颜色,然后将该球放回,并添加一个相同颜色的球到罐中。假设初始时罐中只有一个白球和一个黑球,为了方便,我们让轮次从 $2$ 开始计数,这样第 $n$ 轮后罐中总共正好有 $n$ 个球。令 $X_n$ 表示第 $n$ 轮后黑球的数量,$Z_n = \frac{X_n}{n}$ 表示第 $n$ 轮后黑球所占的比例。显然有 $Z_2 = \frac{1}{2}$。对于所有 $n\ge 2$,
\[
\E{Z_{n+1} \mid \ol{X}_{2,n}} = \frac{1}{n+1} \E{X_{n+1} \mid \ol{X}_{2,n}} = \frac{1}{n+1} \left(Z_n (X_n + 1) + (1 - Z_n) X_n\right) = Z_n.
\]

因此,$\{Z_n\}_{n \geq 2}$ 是相对于 $\{X_n\}_{n \geq 2}$ 的鞅。
\end{example}
\begin{example}[Doob 鞅]

有一类一般的构造鞅的方法叫做 \href{https://en.wikipedia.org/wiki/Doob_martingale}{\underline{Doob 鞅}}。它在随机过程的研究中有非常重要的应用,我们这门课并不会仔细讨论,但我觉得大家在未来的某一天会遇到。

给定一个 Borel 函数 $f\colon \bb R^n\to \bb R$,以及 $n$ 个随机变量 $X_1,\dots,X_n$ ( $n\in \bb N\cup\set{\infty}$,但我们接下来为了方便就假设 $n$ 是自然数)。对于 $k=0,1\dots, n$,定义 $Z_k = \E{f\mid \ol{X}_{1,k}}$。对于 $k>n$,定义 $Z_k = Z_n$。那么 $\set{Z_k}_{k\ge 0}$ 是相对于 $\set{X_k}_{k\ge 1}$ 的鞅,被称为 Doob 鞅。证明很简单,我们只需要对 $k=0,1,2,\dots,n-1$ 验证定义即可。使用条件期望的 Tower Rule:$\forall k=0,1,\dots,n-1$,
\[
\begin{aligned}
\E{Z_{k+1}\mid \ol{X}_{1,k}} 
&= \E{\E{f(X_1,\dots,X_n)\mid \ol{X}_{1,k+1}}\mid \ol{X}_{1,k}}\\
&= \E{f(X_1,\dots,X_n)\mid \ol{X}_{1,k}}\\
&= Z_k.
\end{aligned}
\]

这个鞅的定义非常一般且抽象。我们来看一个具体例子。假设 $X_1,\dots,X_n$ 分别对应了一个图片的每一个像素点的颜色。$(X_1,\dots,X_n)\sim\mu$ 是一个随机的图片。而 $f(x_1,\dots,x_n) = \bb I_{[(x_1,\dots,x_n)对应的图片是一只猫]}$。那么 $Z_0 = \E{f} = \Pr{图片是一只猫}$ 指的是我们完全没有看到这张图的时候该图片是一只猫的概率。而对于 $1\le k\le n$,$Z_k = \Pr{图片是一只猫\mid (X_1,\dots,X_k)}$ 指的是我们看到了前 $k$ 个像素之后,我们对于这个图片是不是猫的猜测概率。特别的,当 $k=n$ 的时候,我们已经不需要猜测,因为 $Z_n = f(X_1,\dots,X_n)$ 是 $\sigma(X_1,\dots,X_n)$-可测的,它的值要么是 $1$ 要么是 $0$,取决于给定的 $(X_1,\dots,X_n)$ 是不是一只猫。
\end{example}